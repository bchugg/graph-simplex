\chapter{Conclusion}
\label{chap:conclusion}

\chapterquote{One has to belong to the intelligentsia to believe things like that: no ordinary man could be such a fool.}{George Orwell, Notes on Nationalism}
\chapterquote{One must imagine Sisyphus happy.}{Albert Camus, The Myth of Sisyphus.}


This dissertation has expounded  and  expanded upon the graph-simplex correspondence, a mapping  between graphs and simplices  first uncovered by Miroslav Fiedler. 
We expanded upon his results in various ways. First, we extended the correspondence to the normalized Laplacian, rather than simply the combinatorial Laplacian. We investigated  the properties of the simplices given by the normalized Laplacian 


\section{Open Problems and Future Directions}
\label{sec:open_problems}

We believe there are several exciting  avenues for further  research. 
\begin{enumerate}
	\item In Section~\ref{sec:algorithmics_complexity} we gave several  examples of how various graph  theoretic problems translate to the simplex and vice  versa, and examined what implications this had for computational complexity. Due to time and space constraints we were unable to fully explore this area; and it seems likely that we have left many  results untapped.  For example, we mostly explored how specific \NP-complete graph problems translated to \NP-complete polyhedral problems. It could be fruitful to explore the converse. More importantly for possible applications, problems which are easy (meaning, solvable in polynomial time) in one domain may have analogues  in the other, which would result in new polynomial  time  algorithms. 
	\item Embedding the approximate distance matrix
	\item Generating an explicit equation for the dual of $\splxn_G$ and $\splx_G^+$. 
\end{enumerate}

