\chapter{Conclusion}
\label{chap:conclusion}

%\chapterquote{One has to belong to the intelligentsia to believe things like that: no ordinary man could be such a fool.}{George Orwell, \emph{Notes on Nationalism}}
\chapterquote{One must imagine Sisyphus happy.}{Albert Camus,  \emph{The  Myth of Sisyphus}}


This dissertation has expounded  and  expanded upon the graph-simplex correspondence, a relationship which associates with each connected, weighted graph $G$ four simplices: $\splx_G$, $\splx_G^+$ (the combinatorial simplices), $\splxn_G$, and $\splxn_G^+$ (the normalized simplices).  
Presenting and building on  the previous work of Fiedler~\cite{fiedler1993geometric,fiedler2011matrices} and Devriendt and Van Mieghem~\cite{devriendt2018simplex}, we have seen the synthesis of the geometry of these  simplices with properties of the graph. At a high  level: 
\begin{enumerate}
	\item The geometry of $\splx_G$ is closely related to  the connectivity of $G$, the geometry of  $\splx_G^+$ is related the effective resistance of $G$.  The squared volume of $\splx_G$ is  proportional to  the number (total weight) of  spanning  trees in $G$,  to which  the  squared volume  of $\splx_G^+$ is inversely proportional; 
	\item The volume of the faces of $\splx_G^+$ are closely related to the entries  of the Laplacian matrix and consequently to the length of the vertices of $\splx_G$;
	\item  The Steiner Ellipsoids of $\splx_G$  and $\splx_G^+$ are determined by the eigenvalues of $\L_G$. For any of a  graph's simplices, the ratio   of the volume  of its Steiner  Ellipsoid  to its  own volume is a constant. 
\end{enumerate}
More broadly, we have seen that graphs  and simplices are related by elegant block matrix equations which can  be used to examine the structure  of both objects. The correspondence also provided the insight  used to give a general  formula  for both the volume  of a simplex and its Steiner Ellipsoid in terms of the dual simplex. Additionally, it helped provide  intuition concerning the general  behaviour of  the dual simplex. 

On the more applied side,  we explored the  algorithmic underpinnings  of  the correspondence and established that 
\begin{enumerate}
	\item[3.] transitioning between various objects in the correspondence (exactly) is lower bounded  by the complexity of computing an eigendecompositon; 
	\item[4.] the correspondence can be used to help classify the computational complexity of geometric problems; and 
	\item[5.] there exist low dimensional  embeddings of the simplices which approximately maintain their Gram matrix relations, and low rank  approximations to the Laplacian yielding low dimensional polytopes approximating the geometry  of $\splx_G$ and $\splx_G^+$. 
\end{enumerate}
The main goal,  however,  was not to achieve any particular result but rather to demonstrate the utility of  the graph-simplex correspondence as  a tool with which  to  analyze  graphs and simplices. We hope to have succeeded in our role as evangelist and  convinced the reader to include the correspondence in their mathematical toolkit. We end  by listing several possible directions for future work. 

\section{Open Problems and Future Directions}
\label{sec:open_problems}

We believe there are several exciting  avenues for further  research. 
\begin{itemize}
	\item In Section~\ref{sec:algorithmics_complexity} we gave several  examples of how various graph  theoretic problems translate to the simplex and vice  versa, and examined what implications this had for computational complexity. Due to time and space constraints we were unable to fully explore this area; it seems likely that we have left many  results untapped.  For example, we mostly explored how specific \NP-complete graph problems translated to \NP-complete polyhedral problems. It could be fruitful to explore the converse. More importantly for possible applications, problems which are ``easy'' (polynomial  time solvable) in one domain may have analogues  in the other, which could result in new  efficient algorithms. 
	\item While  we gave implicit conditions on the  dual of $\splxn_G$ and $\splxn_G^+$, we were unable to give their explicit equations. It  would be desirable to discover what these are.   
	\item In Section~\ref{sec:algorithmics_distance_matrix} we presented the problem of embedding an (approximate) distance matrix  in sub-cubic time. This question  seems like an interesting one in general, even without considering our specific motivation. Related to  this is the  connection  between  $\splx_G^+$  and the resistive polytope,  $\re_G$, given in Section~\ref{sec:resistive_polytope}.  Given that $\L_G^+$ is a more widely  studied object than $\splx_G^+$, it's possible that knowledge concerning  the  pseudoinverse can be leveraged  to uncover properties of, or to optimize over,  $\re_G$.  This could translate to similar results  for $\splx_G^+$. 	
	\item One application  of  the correspondence that we explored only briefly  was that of proving the existence of  certain features in simplices and graphs.  It seems possible that there are many  results  along these lines.  For example, Alev \etal recently demonstrated that any graph  can be partitioned into subgraphs such that  each subgraph has a low maximum effective resistance  and only a fraction of the total edges lie between the subgraphs~\cite{alev2017graph}. This demonstrates that the vertices of any hyperacute simplex can be partitioned into  sets  such that the maximum pairwise distance between the vertices in any set is ``small'' and many  vertices  in distinct sets are orthogonal, or approximately so. 

	\item In  a  similar vein, it would be interesting to explore under what conditions  such results generalize  to all simplices. Are there, for instance, necessary and sufficient conditions on when structural  properties of hyperacute simplices  generalize to  all simplices? If so, then when studying such  properties it would be sufficient  to restrict one's attention to inverse simplices of  graphs. 
	
	\item Our study of random walks in  simplices was severely limited  in scope and insight. The natural use of probability distributions over  the nodes as barycentric coordinates,  however, remains intriguing. Additionally,  the connection between  the normalized Laplacian and random walks suggests this  may be a promising approach for generating new insights into the dynamics of random  walks, and stochastic processes on graphs more generally.    
\end{itemize}

Finally, there are two  possible abstractions of the graph-simplex correspondence which  suggest themselves. 

The first comes from considering the natural generalization of simplices to simplicial complexes. A \emph{simplicial complex} is a collection of simplices $\mathfrak{S}$ such that (i) for every $\ssplx\in\mathfrak{S}$, each face of $\ssplx$ is also in $\mathfrak{S}$ and (ii) for all $\ssplx_1,\ssplx_2\in\mathfrak{S}$, $\ssplx_1\cap\ssplx_2$ is a face of  both  $\ssplx_1$ and $\ssplx_2$. It would be interesting to explore whether one can  associate with each simplicial complex a graph or set of graphs. 

The second involves exploring more fully the mapping  we introduced  in  Section~\ref{sec:correspondence_polyhedra_matrices} which associates a polytope of rank  $d$ with each PSD matrix of rank  $d$. Is  such geometry a useful  way of thinking about these matrices? 