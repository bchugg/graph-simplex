\chapter{Algorithmics}
\label{chap:algorithmics}

This final technical chapter will discuss some of the algorithmic foundations and consequences of the graph-simplex correspondence. Vis-\`{a}-vis foundations, we will chiefly be concerned with transitioning between a graph and its various simplices. We will explore lower bounds for how quickly this can be done if we wish to obtain the precise result\footnote{Ignoring issues of floating point number accuracy}, and whether we can ``approximate'' any of the constructions (e.g., given the graph $G$ can we quickly obtain a simplex which serves as an approximation\footnote{The notion of approximating a simplex is rather ambiguous and will be expounded upon at a later time.} to $\splx_G$.) With respect to algorithmic consequences on the other hand, we will attempt to leverage knowledge we have in the hitherto relatively unrelated areas of computational graph theory and high-dimensional computational geometry to draw new conclusions about the complexity of several problems in these areas. For instance, if a  graph theoretic problem has an analogue in the simplex, any fact regarding the problems difficulty---whether it's NP-complete, say---translates to an immediate result about its geometric counterpart. In particular, since the simplex of a graph can be generate in polynomial time given the graph (due to the fact that an eigendecomposition can be computed in polynomial time) and vice versa, problems which are solvable in polynomial in either the simplex or graph domain  translate to polynomial (yet perhaps not optimal!) problems in the other domain and likewise, problems which are \NP-hard in one domain have analogues which are \NP-hard in the other. 

For the benefit of the (undoubtedly confused) reader unfamiliar with computational complexity and reductions, we begin the chapter with a short section containing this background material. We will also discuss computational representations of a simplex therein. 

\section{Preliminaries}
\label{sec:algorithmics_prelims}
\paragraph{Asymptotics.}
We begin with asymptotic notation which will be used to analyze the running time of various algorithms. We use the standard definitions---see any reference text on algorithm design for more background (e.g., \cite{kleinberg2006algorithm}). Let $f,g:U\subset \R \to \R$ be functions. Write $f=O(g)$ (or $f(n)=O(g(n))$) if $\limsup_{x\to \infty}|f(x)/g(x)|<\infty$, and $f=\Omega(g)$ if $g=O(f)$. Write $f=o(g)$ as $x\to c$ if $\lim_{x\to \infty}|f(x)/g(x)|=0$ and $f=\omega(g)$ if $g=o(f)$. If $f=O(g)$ and $f=\Omega(g)$ we write $f=\Theta(g)$. We will also use the tilde to hide polylog factors. Say $f=\tO(g)$ if $f(n) = O(g(n) \log^c n)$ and $f=\tOmega(g)$ if $f(n) = \Omega(g(n) \log^{-c}n)$, for some $c\geq 0$. 


\paragraph{Simplex representations.}
In order to discuss the algorithmics pertaining to simplices and convex polyhedra in general, we must discuss how such objects are represented by a machine. Clearly, we cannot simply enumerate all the points enclosed by a body in high-dimensional space. Instead we must concisely represent the boundaries of the polytope. The two most common such descriptions are 
\begin{itemize}
	\item \emph{\vdesc}, in which we are given the vertex vectors of the polytope; 
	\item \emph{\hdesc}, in which we are given the parameters of the half-spaces whose intersection defines the polytope. That is, if $\ssplx=\bigcap_i \{\x:\la \z_i,\x\ra \geq b_i\}$, then an \hdesc of $\ssplx$ would be the vectors $\{\z_i\}$ and the scalars $\{b_i\}$. 
\end{itemize}

It's not at all clear whether these descriptions are equivalent in the sense that one can easily generate one from the other. Indeed, the complexity of vertex enumeration (generating a \vdesc from an \hdesc) and facet enumeration (generating an \hdesc from a \vdesc) remains an open problem for general polytopes~\cite{kaibel2003some}, although there exist polynomial time algorithms when the polytopes are simplices (e.g., \cite{bremner1998primal}). We will return to this fact later on. 

\paragraph{Reductions.}
Some background on computational models and reductions will also be useful. For more details see~\cite{kleinberg2006algorithm} or \cite{knuth2011art}. We will use the typical computational model for analyzing algorithms. Without diving too far into the minutiae, we assume that single arithmetic operations require $O(1)$-time, i.e., constant. We will analyze the runtime of an algorithm as a function of how many bits it takes to represent the input. A common tool for providing upper bounds on the runtime of an algorithm is to ``reduce'' it to a problem for which a bound is already known. Assume problem $P$ requires time $\Omega(f_P(n))$ to solve---meaning that \emph{any} algorithm requires time $\Omega(f_P(n))$---where $n$ represents the size of the input and $f_P(n)$  is some function of $n$, e.g., $f_P(n) =  n^2\log n$. Let $Q$ be a distinct problem and suppose that for every instance of $P$ we can transform the input of $P$ to a valid input for $Q$, and transform the output of $Q$ to a valid output of $P$, both in time $O(f_P(n))$.  We have then established that $f_Q(n) = \Omega(f_P(n))$, where $f_Q$ the runtime required to solve $Q$, since we can solve $P$ in time $f_Q(n) + O(f_P(n))$ by transforming any input to $P$ to the input of $Q$, solving $Q$, and transforming the output back. Such a technique will be used extensively throughout the next few sections. 

\paragraph{The complexity classes \NP, \NP-hard, and \NP-Complete.}




\section{Computational Complexity}

In this section we investigate the relationships between problems in one domain---either the graph-theoretic or geometric domain---and their analogues in the other. The following result exemplifies the power of the graph-simplex correspondence in yielding results which seem otherwise to be difficult to obtain (certainly more difficult than the following proof, at any rate).  
The following result was first stated by Devriendt and Van Mieghem~\cite{devriendt2018simplex}, although it was stated only for inverse simplices of graphs. We observe that it can be generalized as follows. 

\begin{lemma}
	\label{lem:altitude_hard}
	Computing the altitude of minimum length in a hyperacute polytope is \NP-hard. Consequently, computing the minimum length altitude in general polyhedra is \NP-hard. 
\end{lemma}
\begin{proof}
	The relationship $\norm{\alt(\splx^+_U)}_2^2 = w(\delta U)^{-1}$ (Lemma \ref{lem:alt}) for the inverse simplex of a graph $G$ demonstrates that the problem of computing a minimum length altitude in any hyperacute simplex is \NP-hard, because computing the  maximum weight cut in any weighted graph is \NP-hard~\cite{karp1972reducibility}.  Since the class of convex polytopes contains the class of hyperacute simplices, the result follows. 
\end{proof}

\begin{remark} In  the above statement and its proof, the description of the polytope and simplex was not specified. This is due to the fact that---as discussed above---for simplices there is a polynomial time algorithm to translate betweent the various descriptions. With regard to \NP-completeness therefore, the description makes no difference. 
\end{remark}

\begin{remark}
	As exemplified by the statement of Lemma \ref{lem:altitude_hard} the fact that a problem is \NP-hard for hyperacute simplices immediately implies that it is so for general polyhedra (since simplices are a subclass of polyhedra). We will still, however, often state a result in terms of general polyhedra because it seems most likely to be useful in this form. 
\end{remark}

The remainder of this section is dedicated to obtaining more results of this type. 

We begin by investigating independent sets. Given a graph $G=(V,E,w)$, recall that an \emph{independent set} is a subset $I\subset V$ such that if $i,j\in I$ then $(i,j)\notin E$. 
The weight of an independent set is nicely described by the Laplacian quadratic form. If $I$ is an independent set note that 
\[\vol(I) = w(\delta I),\] 
and so 
\begin{align*}
    \Lf(\bchi_I) = \sum_{i\sim j}w(i,j)(\bchi_I(i)-\bchi_I(j))^2 = \sum_{i\in I}\sum_{j:j\sim i} w(i,j) = \sum_{i\in I}w(i)=w(\delta I),
\end{align*}
where the second and fourth inequalities follows from the fact that $I$ is an independent set. Now, suppose we assign each vertex $i$ a weight $f(i)\geq 0$. The \mwis problem consists of maximizing $f(I)\equiv \sum_{i\in I}f(i)$ over all independent sets $I$. Clearly \mwis is \NP-hard in general, seeing as it reduces to the usual independent set maximization problem by taking $f(i)=1$ for all $i$. If $f$ is a linear function of the weights so that $f(i)=\alpha w(i)$ for all $i$ and some $\alpha> 0$, we call the corresponding problem $\alpha$-\vwis. We will focus on the case $\alpha=1$ for clarity, and call the  corresponding problem just \vwis. The difficulty of this problem is not immediately clear, since it is more structured than simply \mwis. The next lemma removes any doubt as to the problems tractability.   

\begin{lemma}
	\label{lem:vwis}
	\vwis is \NP-Complete. 
\end{lemma}
\begin{proof}
	Given a purported independent $I$, it is easily checkable in polynomial time whether $\vol(I)$ is of a certain size---hence \vwis is in \NP. 
	To that it is \NP-hard, we reduce from \iset. Let $G=(V(G),E(G))$ and  $k\in\N$ be an instance of \iset. The intuition behind the following reduction is to create a separate graph $H$ which, for each independent set $I\subset V(G)$, has an independent set $J$ in $H$ such that  $\vol_H(J)=|I|$ in $H$ and conversely, for each maximal independent set $J$ in $H$ there exists an independent set $I$ in $G$ with $|I| = \vol_H(J)$. From this relationship it follows that $G,k$ constitutes a yes instance to \iset iff $H,k$ is a yes instance to \mwis.  After wordsmithing the intuition, let us proceed to the formal argument. 

	Construct a graph $H=(V(H), E(H))$ as follows. For each vertex $u\in V(G)$, create $\deg_G(u)+1$ vertices $u_0,u_1,\dots,u_{\deg_G(u)}$ in $V(H)$. For $1\leq k\leq \deg_G(u)$ set \[w_H(u_k) =  \frac{1}{\deg_G(u)}.\] Construct the edge set $E(H)$ such that the neighbours of each vertex are described by 
	\begin{equation*}
	\delta_{H}(u_k) = \{u_0\}\cup \bigcup_{v\in \delta_G(u)}\{v_\ell: 0\leq \ell\leq \deg_G(v) \}.
 	\end{equation*}
	In words, $u_k$ is connected to all the vertices representing $v$ if $(u,v)\in E(G)$, and to $u_0$. Now, let $I\subset V(G)$ be an independent set in $G$ and consider the set
	\[J = \{v_k: v\in I, 1\leq k\leq \deg_G(v)\}.\]
	We claim that $J$ is an independent set in $H$. 
	Indeed, if $v_k,u_\ell\in J$ and  $(v_k,u_\ell) \in E(H)$ for some $k\in[\deg_G(v)]$, $\ell\in[\deg_G(u)]$ then $v\in d_G(u)$ by definition of $\delta_H(u)$. Since $I$ is an independent set however, both $u$ and $v$ are not in $I$, a contradiction. This demonstrates that $J$ is bonafide independent set. Moreover, 
	\[\vol_H(J) = \sum_{v\in I}\sum_{k=1}^{\deg_G(u)} w_H(v_k) = \sum_{v\in I}\sum_{k=1}^{\deg_G(u)} \frac{1}{\deg_G(u)} = |I|.\]

	Conversely, let $J$ be an independent set in $H$. We claim that there exists an independent $J'$ in $H$ with $\vol_H(J')\geq \vol_H(J)$ containing only vertices of the form $v_\ell$ for $\ell\geq 1$, i.e., not $v_0$. Initially, set $J'=J$ but suppose $v_0\in J$. Replace $v_0$ by $v_1,\dots,v_{\deg_G(v)}$ in $J'$.  None of the these vertices share edges, and aside from one another, $v_\ell$ and $v_0$ for $\ell>0$ have the same edge set. It follows that $J'$ remains an independent set. Moreover, since $w_H(v_0) < w_H(v_\ell)$ by construction, we have $\vol_H(J)< \vol_H(J')$. Let us remark further that if $J$ contains vertices $\{v_\ell\}_{\ell\in F}$ for some $F\subsetneq  [\deg_G(v)]$, then we may add the missing vertices $v_k$, $k\in [\deg_G(v)]\setminus F$ while maintaining the property that $J$ is an independent set (this follows since $\delta_H(v_k) = \delta_H(v_\ell)$ for all  $\ell,k\geq1$). We have thus argued that every maximal independent set in $H$ can be written in the form $J= \cup_{v\in I}\{v_k: 1\leq k\leq \deg_G(v)\}$ for some set $I\subset V(G)$. We now claim that $I$ is an independent set in $G$. The argument is similar to above: If not, then $u,v\in I$ with $u\sim v$, but this implies that $v_k\sim v_\ell$ in $H$ meaning that $J$ is not an independent set.  Additionally, as above, $\vol_H(J)=|I|$. Therefore, there exists an independent set $J$ in $H$ with $\vol_H(J)\geq k$ iff there exists an independent set $I$ in $G$ with $|I|\geq k$, concluding the argument. 
\end{proof}

This result allows us to conclude that certain optimizations problems in hyperacute simplices---thus convex polytopes in general---are \NP-hard. 

\begin{lemma}
	Let $\P$ be a convex polytope with vertex set $V$. The optimization problem 
	\begin{alignat*}{2}
	\min_{I\subset V, I\neq\emptyset} & \quad &&  \frac{\norm{\cent(\P_I)}_2^2}{|I|} \\
	\text{s.t.}&  &&  \la \sv_i,\sv_j\ra=0,\; i,j\in I,
	\end{alignat*}
	is \NP-hard. In particular, it is \NP-hard whenever $\P$ is the combinatorial simplex of a graph. 
\end{lemma}
\begin{proof}
	Let $\P$ be the combinatorial simplex of a graph $G$. Using that $\la \sv_i,\sv_j\ra = w(i,j)$, the condition that $\la \sv_i,\sv_j\ra =0$ for all $i,j\in I$ translates to $(i,j)\in E(G)$ for all $i,j\in I$. Moreover, Equation \eqref{eq:c(S_U)} in Section \ref{sec:S_G} gives us  
	\begin{align*}
	\frac{|I|}{\norm{c(\splx_I)}_2^2} = w_G(\delta I) = \vol(I),
	\end{align*}
	for $I$ an independent set. 
	The above optimization problem can consequently be formulated as 
	\[\max_{I\subset V(G)} \vol_G(I),\quad  \text{s.t.} \quad I \text{ is an independent set}.\]
	which is precisely the \vwis problem. 
	\end{proof}
	
We can play a similar game by using the relationships furnished by the normalized Laplacian as opposed to the combinatorial Laplacian. Doing this removes the normalizing factor of $|I|$ from the optimization problem in the previous result.  

\begin{lemma}
	Let $\P$ be a convex polytope with vertex set $V$. The optimization problem
	\begin{alignat*}{2}
	\min_{I\subset V, I\neq \emptyset} & \quad &&  {\norm{\cent(\P_I)}_2^2} \\
	\text{s.t.}&  &&  \la \sv_i,\sv_j\ra=0,\; i,j\in I,
	\end{alignat*}
	is \NP-hard. In particular, it is hard for those polytopes and simplices with all vertices on the unit sphere. 
\end{lemma}
\begin{proof}
	The proof is similar to the previous lemma. For $\P$ the normalized simplex of a graph $G$, the condition $\la \sv_i,\sv_j\ra=0$ once again implies that $I$ must be an independent set. Notice that for such an $I$, if $i\in I$ then $\delta(i) \cap I^c = \delta(i)$ (none of $i$'s neighbours are in $I$). Therefore, Equation \eqref{eq:Lnf(chiU)} yields 
	\begin{align*}
	\Lnf(\bchi_I)=\sum_{i\in I}\frac{1}{w(i)}\sum_{j\in I^c\cap \delta(i)} w(i,j) = \sum_{i\in I}\frac{w(i)}{w(i)} = |I|.
	\end{align*}
	Equation \eqref{eq:c(SnU)} then implies that 
	\[\norm{c(\P_I)}_2^2=\frac{1}{|I|^2}\Lnf_G(\bchi_I)=\frac{1}{|I|},\]
	so the optimization problem can be formulated as 
	\[\max_{I\subset V(G)} |I|,\quad  \text{s.t.} \quad I \text{ is an independent set},\]
	which is the \iset problem. 
\end{proof}


Next we extract a result based on the most (in)famous problem in computational graph theory: Graph isomorphism. An \emph{isomorphism} between two graphs $G_1$ and $G_2$ is a bijection $f:V(G_1)\to V(G_2)$ such that $(u,v)\in E(G_1)$ iff $(f(u),f(v))\in E(G_2)$. We write $G_1\cong G_2$ if $G_1$ is isomorphic to $G_2$. The \graphiso problem asks, given $G_1,G_2$ whether they are isomorphic. 
It's clear that $\graphiso\in\NP$, but whether it is \NP-complete remains an open question~\cite{mckay2014practical}. L{\'a}szl{\'o} Babai recently claims to have solved the problem in  quasipolynomial time~\cite{babai2016graph}; the work is still being verified. 
Accordingly, we call a problem Graph-Isomorphism-Hard if it can be reduced to to \graphiso. 
The more general problem of \emph{subgraph} isomorphism, which asks whether $G_1$ has a subgraph isomorphic to $G_2$, is \NP-complete~\cite{cook1971complexity, karp1972reducibility}. We are interested in the relationship between graph isomorphism and polytope congruence.  

\begin{theorem}
	\label{thm:simplex_congruence}
Deciding whether two hyperacute simplices are congruent is Graph-Isomorphism-Hard. Moreover, given two hyperacute simplices $\splx_1\in\R^d$ and $\splx_2\in\R^k$, deciding whether there exists $k$-dimensional face of $\splx_1$ congruent to $\splx_2$ is \NP-hard. 
\end{theorem}
\begin{proof}
	Let two graphs $G_1$ and $G_2$ be given. Compute their corresponding inverse simplices $\splx_1^+$ and $\splx_2^+$. 
	We claim that $\splx_1^+\cong\splx_2^+$ iff $G_1\cong G_2$. If $\splx_1^+\cong\splx_2^+$ then because they are both centred at the origin there exists a rotation matrix $\Q$ such that $\Q\Sv_1^+=\Sv_2^+$. Since a rotation matrix does not change the relationship between the inner product of vectors\footnote{A rotation matrix $\Q$ obeys $\Q^\tp\Q=\I$, hence $\la \Q\u,\Q\v\ra = \u^\tp \Q^\tp \Q\v=\la \u,\v\ra$.}, we see that $(\Sv_1^+)^\tp\Sv_1^+$ and $(\Sv_2^+)^\tp\Sv_2^+$ define the same Laplacian. Hence $G_1$ is isomorphic to $G_2$. Conversely, if $G_1\cong G_2$, then there exists a  relabelling of the vertices such that their Laplacian matrices are identical, as are the simplices. The second part of the statement follows by a similar reduction, and the fact that $\subgraphiso\in\NP$-complete.
\end{proof}

Kaibel and Schwarz~\cite{kaibel2008complexity} investigated the problem of polytope isomorphism. They define two polytopes is isomorphic if they have the same \emph{face-lattice}---the lattice in which the nodes correspond to subsets of the vertices, and the lattice ordering is by face inclusion. Since congruent simplices share the same face  lattice up to labelling, Theorem \ref{thm:simplex_congruence} implies their result. 


\section{There and Back Again: A Tale of Graphs and Simplices}
In this section we investigate the computational aspects of transitioning between the various objects which we've studied thus far. As one should expect given that the mapping between graphs and simplices relies on the  eigenvalues and eigenvectors of graph  Laplacians, the complexity of these transitions is intimately related with the complexity of computing  eigendecompositions. 
Moreover, as we will see, if we are prepared to compute  eigendecompositions (which is essentially cubic), then we can essentially compute all the objects from one another. We thus begin  with a foray into the computational complexity of eigendecompositions, as we will be mostly  interested in circumstances in which a transition can be computed in less  time than this. Unfortunately, it will become clear that the complexity of  computing a Laplacian eigendecomposition is actually a lower bound to many of the transitions. 
 
Let $M(n)$ denote the complexity of the eigendecomposition problem. It is known that  $M(n)=\tOmega(n^3 + n\log^2 \log \eps)$ to obtain a relative error\footnote{We note that the relative error is a necessary parameter of any algorithm because eigenvalues may be irrational.} of $2^{-\eps}$, while there exists algorithms which run in time $O(n^3 + n\log^2 \log \eps)$~\cite{pan1999complexity}.  
Let \lapdecomp refer to the problem of computing the eigendecomposition of the Laplacian of a graph, i.e., computing its eigenvalues and eigenvectors. The complexity of \lapdecomp does not seem to be known, \note{really need to figure this out---how can it not be known?} and we thus denote the lower bound by $\Omega(n^\tau)$ for some $\tau$. We will assume, based on the difficulty of general eigendecomposition that $\tau>2$. 


Now, observe that given $G$, we can compute the combinatorial and nornalized Laplacians (and their inverses) by first constructing the combinatorial or normalized Laplacian in $O(n^2)$, performing an eigendecomposition in time $O(n^\tau)$, and constructing the vertices of the simplices from the eigenvalues and eigenvectors in time $O(n^2)$. Using our  assumption that $\tau>2$, this takes total time $O(n^\tau)$.  Moreover, starting with a simplex with vertex set $\Sv$, one can compute $\Sv^\tp\Sv$ in the time required for matrix multiplication, which is currently $O(n^{2.3727})$~\cite{williams2012multiplying} and whose lower bound is $\Theta(n^\kappa)$ for some $2\leq \kappa\leq 2.3727$~\cite{stothers2010complexity}. If the simplex is the simplex of a graph then this yields the Laplacian (or its pseudoinverse) of the graph in time $O(n^{2.3727})$,  and from here to any of its simplices  in time $O(n^\tau)$. Hence, we can transition between the various simplices in time $O(n^{\max\{2.3727,\tau\}})$.  In what follows therefore, we attempt to beat the barrier of $O(n^\tau)$. 

Another question in which we might be interested is one of \emph{certification}. That is, verifying whether  a given simplex is one of the combinatorial or normalized simplices of a graph. We will investigate  this possibility at  the end of this  section. 

We begin by investigating the relationship between $\splx$ and $\splxn$, when either  $\splx$ or $\splxn$ are given and we are told a priori that they are the simplices of a graph. The results obtained  in this section are summarized in Figure~\ref{fig:mapping_results}. 

\begin{figure}
	\centering
	\renewcommand{\arraystretch}{1.5}	\begin{tabular}{|c|c|c|c|c|c|c|c|c|c|c|}
		\hline 
		\multicolumn{3}{|c|}{} & \multicolumn{4}{c|}{\textsf{V}} & \multicolumn{4}{c|}{\textsf{H}}\\
	\cline{3-11} 
\multicolumn{2}{|r|}{From/To} & $G$ & $\splx_G$ & $\splx_G^+$ & $\splxn_G$ & $\splxn_G^+$ & $\splx_G$ & $\splx_G^+$ & $\splxn_G$ & $\splxn_G^+$ \\
\cline{2-11} 
& $G$ & --- &$\Omega(n^\tau)$ &$\Omega(n^\tau)$ &$\Omega(n^\tau)$ &$\Omega(n^\tau)$ & $\Omega(n^\tau)$ & $\Omega(n^\tau)$ &  & \\
\hline 
\multirow{4}{0.4cm}{\textsf{V}} & $\splx_G$ & $O(n^3)$ & --- & $\Omega(n^\tau)$ & $O(n^2)$ & & $\Omega(n^\tau)$ & $O(1)$ & & \\
\cline{2-11}
& $\splx_G^+$ & & $\Omega(n^\tau)$ & --- & & & $O(1)$ &$\Omega(n^\tau)$ & & \\
\cline{2-11}
& $\splxn_G$ & & ?  / $O(n^2)$ &  & --- & $\Omega(n^\tau)$ & & & &\\
\cline{2-11}
& $\splxn_G^+$ & & & &$\Omega(n^\tau)$ & --- & & & & \\
\hline 
\multirow{4}{0.4cm}{\textsf{H}} & $\splx_G$ & & $\Omega(n^\tau)$ & $O(n^2)$ & & &--- & $\Omega(n^\tau)$& & \\
\cline{2-11}
& $\splx_G^+$ & & $O(n^2)$ & $\Omega(n^\tau)$ & & &$\Omega(n^\tau)$ & --- & & \\
\cline{2-11}
& $\splxn_G$ & & & & &  & & & --- &\\
\cline{2-11}
& $\splxn_G^+$ & & & & & & & & & --- \\
\hline 
	\end{tabular}
	\renewcommand{\arraystretch}{1}
\caption{Summary of results for precise mappings. A slash refers to a difference in runtimes when the graph is available versus when it isn't. The quantity before the slash indicates the runtime \emph{without} the graph, after the slash the runtime \emph{with} the graph. A question mark indicates that the runtime isn't known.}
\label{fig:mapping_results}
\end{figure}

\paragraph{Between \texorpdfstring{$\splx$}{the combinatorial} and \texorpdfstring{$\splxn$}{normalized simplex}.}
Let us consider the computational complexity of transitioning between $\splx$ and $\splxn$ and vice versa. Let $\phi_{ij}$ (resp., $\phin_{ij}$) be the angle between $\sv_i$ and $\sv_j$ (resp., $\svn_i$ and $\svn_j$). Using the typical formula for the dot product in Euclidean space we have
\begin{equation*}
\cos\phi_{ij} = \frac{\la \sv_i,\sv_j\ra }{\norm{\sv_i}_2\norm{\sv_j}_2} = \frac{\L_G(i,j)}{\sqrt{w(i)w(j)}} = \Ln_G(i,j), \quad\text{and}\quad \cos\phin_{ij} = \frac{\la \svn_i,\svn_j\ra }{\norm{\svn_i}_2\norm{\svn_j}_2} = \Ln_G(i,j),
\end{equation*}
using that $\norm{\svn_i}_2=1$ for all $i$. 
That is, the angles between vertices in $\splx$ in $\splxn$ are the same. Suppose we are given the simplex $\splx$ and told it is the combinatorial simplex of a graph. For each $\sv_i = \Sv(\splx)$, define a new vertex 
\[\bgamma_i = \frac{\sv_i}{\norm{\sv_i}_2}.\]
Is it evident that the angle between $\bgamma_i$ and $\bgamma_j$ is identical to that between $\sv_i$ and $\sv_j$: 
\begin{equation*}
\frac{\la \bgamma_i,\bgamma_j\ra}{\norm{\bgamma_i}_2\norm{\bgamma_j}_2} = \bigg\la \frac{\sv_i}{\norm{\sv_i}_2},\frac{\sv_j}{\norm{\sv_j}_2}\bigg\ra = \cos(\phi_{ij}).
\end{equation*}
Therefore, it follows that the simplex with vertices is congruent  to $\splxn$. This yields the following result. 

\begin{lemma}
	Given a combinatorial simplex $\splx$, a simplex congruent to $\splxn$ can be computed in time $O(n^2)$. 
\end{lemma}
\begin{proof}
	Given $\splx$, define the vertices $\bgamma_i$ as above. Computing $\norm{\sv_i}_2$ takes time $O(n)$ and must be done for each vertex. 
\end{proof}

Given the relative ease with which we can transition from $\splx$ to $\splxn$, it is somewhat surprising that it is much more difficult to transition from $\splxn$ to  $\splx$, especially if the underlying graph $G$ is not given. The obvious tactic is, given the vertices $\{\svn_i\}$, to define vertices $\svn_i \sqrt{w(i)}$, which, since $\sqrt{w(i)}=\norm{\sv_i}_2$, have the same magnitude as $\sv_i$. As above, the scaling does not affect the angle between the vertices, and thus the simplex with these vertices is congruent to $\splx$. However, it's not clear how to obtain the value $\sqrt{w(i)}$ from $\splxn$. Using that $\la \svn_i,\svn_j\ra =(w(i)w(j))^{-1/2}$ we can write 
\[w(i)^{1/2} = -\sum_{j\neq i}w(j)^{-1/2} \bigg/ \sum_{j\neq i}\la \svn_i,\svn_j\ra,\]
which yields a non-linear system of equations. 

Of course, if we are given the graph then we have access to $\sqrt{w(i)}$ and can compute $\svn_i w(i)^{1/2}$ in time $O(n)$. The following result is then immediate. 

\begin{lemma}
	Given a graph $G=(V,E,w)$ and its normalized simplex $\splxn_G$, a simplex congruent to  the combinatorial simplex $\splx_G$ can be computed in $O(n^2)$ time. 
\end{lemma}

\note{Think about possible lower bounds on computing $\splx$ from $\splxn$ when no graph is given. Doing so would imply knowledge of $\sqrt{\w}$ (taking ratio of lengths of vertices). What does this imply? Does knowledge of $\w$ give us some knowledge of the graph structure from which we can extract a lower bound? }



\paragraph{\texorpdfstring{$\splx$}{The combinatorial} and \texorpdfstring{$\splx^+$}{normalized simplex}.}

Let us suppose that we can generate $\splx^+$ from $\splx$ (or vice versa) in time $O(g(n))$. Note that for $i<n$, 
\[\lambda_i = \frac{\lambda_i^{1/2} \vp_j(i)}{\lambda_i^{-1/2}\vp_j(i)} = \frac{\sv_i(j)}{\sv_i^+(j)}, \quad \text{and} \quad \vp_i(j) = \frac{\sv_j(i)}{\lambda_i^{1/2}},\]
hence knowledge of $\{\sv_i\}$ and $\{\sv_i^+\}$ yields knowledge of the eigendecomposition of the underlying graph $G$ in $O(n^2)$ time ($O(n)$ to determine all the  eigenvalues and $O(n^2)$ to determine the eigenvectors). The same argument holds \emph{mutatis mutandis} for the normalized Laplacian. 

\begin{lemma}
	\label{lem:S_to_S^+_vdesc}
	If a \vdesc of $\splx^+$  (resp., $\splxn^+$) can be generated from a \vdesc of $\splx$ (resp., $\splxn$) or vice versa in time $O(g(n))$, then \lapdecomp can be solved in time $O(g(n) + n^2)$ for arbitrary weighted graphs. Consequently $g(n) = \Omega(n^\tau)$.  
\end{lemma}

An alternate way of seeing that constructing the inverse simplex from its dual is computationally challenging is to recall from Section \ref{sec:S_G} that $\splx_\ic$ is contained in the hyperplane $\{\x\in\R^{n-1}:\la \x,\sv_i^+\ra = -1/n\}$ (Lemma \ref{lem:SUsubset})
 and that that $\sv_i^+$ is perpendicular to $\splx_\ic$ (Lemma \ref{lem:S_G_basic_properties}). Hence, computing the inverse simplex would imply that we had computed normal vectors to $n$ hyperplanes, the typical procedure for which typically involves computing an $n\times n$ determinant and requires  $O(n^3)$ time. 
 
 We now consider transitioning between different descriptions of $\splx$ and $\splx^+$. Let us recall that the \hdesc of $\splx$ and $\splx^+$ yield immediate insight into the vertices of its inverse as $\splx=\cap_i \{\x:\la \x,\sv_i^+\ra \geq -1/n\}$ and $\splx^+=\cap_i\{\x:\la\x,\sv_i\ra \geq -1/n\}$ (Equations \eqref{eq:splx_bigcapH_i} and \eqref{eq:splx^+_bigcapH_i}). Consequently, given given a \hdesc of one of these simplices, the vertices of its inverse are recoverable in quadratic time. This yields the following result. 
 
 \begin{lemma}
 	\label{lem:S_vdesc_to_hdesc}
	 Suppose we can compute an \hdesc of $\splx$ (resp., $\splx^+$) )given its \vdesc in time $t(n)$. Then a \vdesc of $\splx^+$ (resp., $\splx$) is recoverable in time $t(n) + O(n^2)$, implying by Lemma \ref{lem:S_to_S^+_vdesc} that $t(n) = \Omega(n^\tau)$. 
 \end{lemma}

We also note that a consequence of the relationship between the vertices of $\splx$ and the \hdesc of $\splx^+$ that given \vdesc of $\splx$ or $\splx^+$, we have immediate access to the \hdesc of its inverse. 

A similar result for going from between the \hdesc of the combinatorial simplices. The argument runs as usual: Given an \hdesc of $\splx$, suppose we can generate an \hdesc of $\splx^+$ in time $t(n)$. We can obtain the vertices $\{\sv_i^+\}$ from the \hdesc of $\splx$, and the vertices $\{\sv_i\}$ from the \hdesc of $\splx^+$. Using these, we can then obtain the eigendecomposition of $G$ in time $O(n^2)$. That is, we can solve \lapdecomp in time $t(n) + O(n^2)$ yielding that $t(n) = \Omega(n^\tau)$. 

\begin{lemma}
	\label{lem:hdesc_to_hdesc}
	Generating an \hdesc of $\splx_G$ given an \hdesc of $\splx_G^+$, and vice versa, requires time $\Omega(n^\tau)$. 
\end{lemma}


\paragraph{Between \texorpdfstring{$G$}{the graph} and \texorpdfstring{$\splx$ or $\splxn$}{its simplices}.}
Similar kinds of results hold  in these cases. Assume that we obtain  the simplex $\splx_G$ from $G$. Notice  that \[\sum_{i=1}^{n-1}   \sv_i(j)^2 = \lambda_j \sum_{i=1}^{n-1} \vp_j(i) = \lambda_j\bigg(1-\frac{1}{n}\bigg),\]
so 
\[\lambda_j = \frac{\sum_{i=1}^{n-1}\sv_i(j)}{1-1/n},\]
which can be computed  in $O(n)$  time. Then, as above, knowledge of the eigenvalues furnishes knowledge  of the eigenvectors in $O(n^2)$ time. Running almost identical arguments for $\splx^+$, $\splxn$, or $\splxn^+$ yields an almost equivalent result as in the previous section.  

\begin{lemma}
	\label{lem:G_to_S_and_Sn}
	If either the combinatorial or normalized simplex or their  inverses can be generated from a graph $G$ in $O(g(n))$ time, then \lapdecomp can be solved in time $O(g(n) + n^2)$ for arbitrary weighted graphs. Consequently $g(n) = \Omega(n^\tau)$. 
\end{lemma}

The information encoded in the dot products between vertices allow us to make queries regarding the edge weights, but  each query takes $O(n)$ time since we must compute a dot product between two vectors of length $n-1$. Hence, re-constructing the graph or its Laplacian takes $O(n^3)$ if we wish do it precisely. 

Let us now consider transitioning between $G$ and the \hdesc of a simplex.  The following lemma summarizes the consequences of this relationship. 

\begin{lemma}
	Given a graph $G$ suppose an \hdesc of $\splx$ (resp., $\splx^+$) can be generated in time $g(n)$. Then a \vdesc of $\splx^+$ (resp., $\splx$ can be obtained in time $O(g(n) + n^2)$ starting from $G$. Consequently, by Lemma \ref{lem:G_to_S_and_Sn}, $g(n)=\Omega(n^\tau)$. 
\end{lemma}


\paragraph{Between different descriptions of the simplices.}

Here we investigate  the interplay between the various different descriptions of the simplices. 


The following is an immediate consequence of Lemma \ref{lem:hdesc_dual}.  

\begin{corollary}
	\label{cor:hdesc_S_to_S+}
	If $\ssplx$ is a centred simplex in \hdesc, we can obtain a \vdesc of $\ssplx^D$ in quadratic time. In particular, given  an \hdesc of the combinatorial simplex $\splx_G$ (resp., inverse combinatorial  simplex $\splx_G^+$)  of a graph $G$, a \vdesc of $\splx_G^+$ (resp., $\splx_G$)  is obtainable in quadratic time. 
\end{corollary}

Due to the fact that $\splxn_G^+$ is not the dual of $\splxn_G$ Lemma \ref{lem:hdesc_dual} is less useful here.  

\begin{lemma}
	\label{lem:hdesc_to_vdesc}
	Generating an \vdesc of the simplex $\splx$ given its \hdesc requires time $\Omega(n^\tau)$ for any $\splx\in\{\splx_G,\splx_G^+\}$. 
\end{lemma}
\begin{proof}
	Consider $\splx_G$; the argument is similar for $\splx_G^+$. Suppose obtaining the \hdesc takes time $t(n)$. Due to the properties of the hyperplane representations, this yields access to both sets of vertices in time $t(n)+O(n^2)$. Using the arithmetic in the previous section, this implies that we can obtain the eigenvalues and eigenvectors of $G$ in time $O(n^2)$, i.e., we can solve \lapdecomp in time $t(n)+O(n^2)$ implying that $t(n)=\Omega(n^\tau)$. 
\end{proof}


\paragraph{Verification.}
In time $O(n^{2.3727})$ we can compute $\Sv^\tp\Sv$. We can check whether this is equal to $\L_G$ for some $G$ by verifying whether (i) $\Sv^\tp\Sv\one=\zero$, (ii) $(\Sv^\tp\Sv)(i,i)>0$ for all $i$ and (iii) $(\Sv^\tp\Sv)(i,j)\leq 0$ for all $i\neq j$. These three steps require time $O(n^3)$. We can check whether $\Sv^\tp\Sv$ is equal to $\Ln_G$ for some $G$ by first ensuring, similarly to above, that (iii) holds and that $(\Sv^\tp\Sv)(i,i)=1$ for all $i$. Then we compute the kernel  of $\Sv^\tp\Sv$ in cubic time by means of Gaussian elimination~\cite{kailath1999fast} to obtain a vector $\v$ equal to $\sqrt{\w_G}$ (if indeed $\Sv^\tp\Sv=\Ln_G$) up to scaling. To determine whether $\v$ does represent valid weightings of the vertices, we check whether $(\Sv^\tp\Sv)(i,j)\v(j)$ is constant for all $i$. In  this case $\Sv^\tp\Sv$ is equal to the normalized Laplacian of some graph. This can also be done in cubic time. Therefore, 

Moreover, in cubic time we can check whether all the angles $\theta_{ij}$ between the faces $\ssplx_\ic$ and $\ssplx_\jc$ are non-obtuse, in which case $\ssplx$ is the inverse simplex of some graph. Beyond computing $\Sv(\ssplx)^\tp \Sv(\ssplx)$ however, it's not clear how to obtain the original graph or the combinatorial simplex. 






\section{Approximations} 
\label{sec:algorithmics_approximations}
Here we are concerned  with  approximations of various sorts. We begin  with an eye towards the problem of  dimensionality. Specifically, Theorem ~\ref{thm:graph-simplex} yields simplices of dimension $n-1$ for a graph on $n$ vertices. In  many application areas, graphs may have thousands  to  millions of vertices. Working in a Euclidean space  of  this  size can be unwieldy. Our  first  result, therefore, demonstrates that we can ``approximate'' the simplex in a lower dimensional space. 

\subsection{Embedding \texorpdfstring{$\splx$}{the simplex} in lower dimensions}
The idea is to  map each vertex to a  point in $\R^d$, for $d\ll n$, while maintaining the general form of the simplex. By this we mean that we'd like the distance between the new  points  to remain approximately  as they  were. If possible, we'd also  like to new, lower  dimensional  object (note that it won't  be  a simplex because there will  be  $n$ points in $ R^d$) to retain some of the properties which  relate it  to the underlying graph. In  particular, we'd like the gram matrix  of the new  points to  approximate the gram  matrix of the original  set  of points. As it turns out,  a  mapping meeting both  of these criteria exists and is computable  in polynomial time. It will   rely on the Johnson-Lindenstrauss Lemma~\cite{johnson1984extensions,dasgupta2003elementary}. 

\begin{theorem}[Johnson-Lindenstrauss Lemma]
	Let $E\subset \R^k$ be a set of $n$ points, for some $k\in\N$. For any $\eps>0$ and $d\geq 8\log(n)\eps^{-2}$ there exists a map $g_\eps:\R^k\to\R^d$ such that 
	\begin{equation*}
	(1-\eps)\norm{\u-\v}_2^2 \leq \norm{g_\eps(\u) - g_\eps(\v)}_2^2 \leq (1+\eps)\norm{\u-\v}_2^2,
	\end{equation*}
	for all $\u,\v\in E$. 
\end{theorem}


Consider inverse simplex for which we have $\norm{\sv_i^+-\sv_j^+}_2^2=r(i,j)$ where $r(i,j)$ is the effective resistance between vertices $i$ and $j$. Add a point $\o$ which is the centroid of these points. Thus $\norm{\sv_i^+-\o}_2^2 = \L_G^+(i,i)$ for all $i$. Note that we can compute this in linear time since 
\[\norm{\sv_i^+-\o}_2^2 = \norm{\sv_i^+}_2^2 = \frac{1}{W(\delta(\{i\}))}=\frac{1}{w(i)}.\]

Applying JL transform to obtain $n+1$ points in $\R^d$, for $d=O(\log(n)/\eps^2)$. Let $f$ be the mapping, e.g., $\sv_i^+$ mapped to $f(\sv_i^+)$. By JL, have 

\[(1-\eps)\norm{\x-\y}_2^2\leq  \norm{f(\x) -f(\y)}_2^2\leq (1+\eps)\norm{\x-\y}_2^2, \]
for all $\x,\y\in \{\sv_1^+,\dots,\sv_n^+,\o\}$. 
Apply a linear transformation to the points so that $f(\o)$ coincides with the origin $\zero\in\R^d$. Note that this does not affect the distances between the points themselves, and does not damage the approximation. Update $f$ to reflect this transformation. Then, 
\[\norm{f(\sv_i^+)}_2^2 = \norm{f(\sv_i^+)-f(\o)}_2^2 = (1+\eps_{i,\o})\norm{\sv_i^+-\o}_2^2 = (1+\eps_{i,\o})\L_G^+(i,i).\]
Hence, 
\begin{align*}
\norm{f(\sv_i^+)-f(\sv_j+)}_2^2 &= \la f(\sv_i^+)-f(\sv_j^+),f(\sv_i^+)-f(\sv_j+)\ra \\
&= \norm{f(\sv_i^+)}_2^2 + \norm{f(\sv_j^+)}_2^2 - 2\la f(\sv_i^+),f(\sv_j^+)\ra,  
\end{align*}
implying that 
\begin{align*}
\la f(\sv_i^+),f(\sv_j^+) \ra &= -\frac{1}{2} \bigg((1+\eps_{i,j})\norm{\sv_i^+-\sv_j^+}_2^2 - (1+\eps_{i,\o})\L_G^+(i,i) - (1+\eps_{j,\o})\L_G^+(j,j)\bigg) \\
&= -\frac{1}{2}((1+\eps_{i,j})r(i,j) - (1+\eps_{i,\o})\L_G^+(i,i) - (1+\eps_{j,\o})\L_G^+(j,j)) \\
&= -\frac{1}{2}((1+\eps_{i,j})(\L_G^+(i,i) - \L_G^+(j,j) - 2\L_G^+(i,j)) \\
&\hspace{2cm}- (1+\eps_{i,\o})\L_G^+(i,i) - (1+\eps_{j,\o})\L_G^+(j,j))\\
&= (1+\eps_{i,j})\L_G^+(i,j) + \varepsilon(i,j),
\end{align*}
where 
\[\varepsilon(i,j)\equiv \frac{1}{2}(\eps_{i,\o}-\eps_{i,j})\L_G^+(i,i) + (\eps_{j,\o}-\eps_{i,j})\L_G^+(i,j),\]
is an error term dictated by $\eps_{i,j}, \eps_{i,\o}$ and $\eps_{j,\o}$. Setting 
\[M\equiv \max_i \L_G^+(i,i),\]
we can bound the error term via repeated applications of the triangle inequality: 
\begin{align*}
|\varepsilon(i,j)|& \leq \frac{1}{2}\bigg(|(\eps_{i,\o}-\eps_{i,j})\L_G^+(i,i)| + |(\eps_{j,\o}-\eps_{i,j})\L_G^+(i,j|\bigg) \\
& \leq \frac{1}{2}\bigg([|\eps_{i,j}|+|\eps_{i,\o}|]\L_G^+(i,i) + [|\eps_{i,j}|+|\eps_{j,\o}|]\L_G^+(j,j)\bigg) \\
&\leq \frac{1}{2} ( 2\eps\L_G^+(i,i) + 2\eps\L_G^+(j,j) ) \leq 2\eps M,
\end{align*}
since $|\eps_{i,j}|, |\eps_{i,\o}|,|\eps_{j,\o}|\leq |\eps|$. Setting $f(\Sv^+) = (f(\sv_1^+),\dots,f(\sv_n^+))\in \R^{d\times n}$, this approximation implies that 
\begin{equation*}
\L_G^+ - O(\eps M)\I \leq f(\Sv^+)^\tp f(\Sv^+) \leq \L_G^+ + O(\eps M)\I. 
\end{equation*}
In other words, we can approximately recover the Gram matrix $\L_G^+=\Sv^+\Sv^+$ using the lower dimensional matrix $f(\Sv^+)$. 

The JL mapping maintains other approximate information of the graph.  For  example,  it is well-known that the effective resistance between two vertices is related to the probability that this edge  is in a random  spanning  tree as 
\begin{equation*}
	\effr(i,j) = \frac{1}{w(i,j)} \Pr_{T\sim \mu}[(i,j)\in T],
\end{equation*}
where $\mu$ is the uniform distribution over all spanning trees~\cite{burton1993local}. 
	




\subsection{Approximating the  distances of \texorpdfstring{$\splx_G^+$}{the inverse simplex}}



\begin{theorem}[\cite{spielman2011graph}]
	For any $\eps>0$ and graph $G=(V,E,w)$, there exists an algorithm which computes a matrix $\widetilde{\Reff}\in\R^{O(\log(n)\eps^{-2})\times n}$ such that 
	\begin{equation*}
	(1-\eps)r(i,j) \leq \norm{\widetilde{\Reff}(\bchi_i-\bchi_j)}_2^2 \leq (1+\eps) r(i,j).
	\end{equation*}
	The algorithm runs in time $\widetilde{O}( |E|\log (r)/\eps^2)$, where 
	\[r=\frac{\max_{i,j}w(i,j)}{\min_{i,j}w(i,j)}.\]
\end{theorem}

Given a graph $G=(V,E,w)$, we can compute all the approximate distances $\norm{\sv_i^+-\sv_j^+}_2^2=r(i,j)$ in time \[\widetilde{O}(|E|\log (r)/\eps^2)+O(|E| \log(n)/\eps^2)=\widetilde{O}(|E|/\eps^2),\]
assuming $r=O(1)$. Note that we can compute a single effective resistance in time $O(\log n/\eps^2)$, since it involves simply computing the $\ell_2$ norm the vector $\widetilde{\Reff}(\bchi_i-\bchi_j)$ which is simply the difference of two columns of $\widetilde{\Reff}$. 










\paragraph{Low Rank Approximation}

\note{Define low rank approximation}
Let us suppose the we have obtained a low rank---$k$, say---approximation of $\L_G$, written $\tL$. We might then ask several questions: 
\begin{enumerate}
	\item Is $\tL$ still a gram matrix? That is, can $\tL$ be written $\tSv^\tp\tSv$ where $\tSv$ is the vertex matrix of some set of points, $P=\{\p_1,\dots,\p_\ell\}$? If so, what is the relationship between $\Sv$ and $\tSv$, where $\Sv=\Sv(\splx_G)$ is the usual vertex matrix of the combinatorial simplex of $G$? If $\tL$ has rank $k$ then $P$ spans a subspace of dimension $k$ and $\conv(P)$ forms a polytope in that space. What is the relationship between the geometry of $\conv(P)$ and $\splx_G$?
	\item Is $\tL$ useful in helping estimate properties of the simplex $\splx_G$? For example, if one could bound the difference in the quadratic products of $\L_G$ and $\tL$, this would imply (via the results in Section \ref{sec:S_G}) that we could estimate many of the properties of $\splx_G$. 
	\end{enumerate}

Of course, we have chosen to work with $\L_G$ and $\splx_G$ for convenience; we could have asked the same questions of $\Ln_G$ and $\splxn_G$. 

Let us examine a specific low rank approximation proposed by Drineas and Mahoney~\cite{drineas2005approximating}, which finds low rank approximations to Gram matrices. We will give a brief overview of their method in general, and then elaborate on how it applies to our case in particular. Let $\M\in\R^{n\times m}$ be a gram matrix. Using the probability distribution $F(i) = \M(i,i)^2 / \tr(\M^2)$ sample $a\leq m$ columns of $\M$ independently at random and with replacement, where $a$ is some given parameter. Let $I\subset[n]$, $|I|\leq a$, be the indices of sampled columns. Let $\bC\in\R^{n\times a}$ be the matrix formed by these columns (that is, $\bC=\M(\cdot,I)$).  Let $\Q$ be the matrix $\M(I,I)\in\R^{a\times a}$, i.e., the submatrix of $M$ with entries corresponding to indices in $I$, and $\Q^+_k$  the optimal rank $k$-approximation to $\Q^+$, the pseudoinverse of $\Q$ (section \ref{sec:background_pseudoinverse}). The low rank approximation to $\M$ is then 
\begin{equation*}
\tM \equiv \bC\M(I,I)_k^+\bC^\tp.
\end{equation*}

\begin{theorem}[\cite{drineas2005approximating}]
	Let $\M$ be a gram matrix and let $\tM$ be as above. Let $\eps>0$,  $k\leq c\in\N$. If $c=\Omega(k/\eps^4)$, then 
	\begin{equation*}
	\norm{\M-\tM}_\kappa \leq \norm{\M-\M_k}_\kappa + \eps \tr(\M^2),
	\end{equation*}
	for $\kappa=2,F$. 
\end{theorem}

Let us analyze how this result translates to the case when $\M=\L_G$. Let $I$ and $\bC$ be as above. First we observe that $\L_G(I,I)$ is simply the Laplacian on the subgraph $G[I]$. Put $\tG = G[I]$. Performing an eigendecomposition, write 
\begin{equation*}
\L_\tG = \sum_{r=1}^{|I|} \mu_r \bnu_r\bnu_r^\tp,
\end{equation*}
for where $\mu_1\geq \mu_2\geq \dots \geq \mu_{|I|}=0$ and $\{\bnu_r\}$ are the eigenvalues and eigenvectors of $\L_\tG$, respectively. The results of Section \ref{sec:background_pseudoinverse} then dictate that 
\begin{equation*}
\L_\tG^+ = \sum_{r=1}^{|I|} \frac{1}{\mu_r} \bnu_r\bnu_r^\tp,
\end{equation*}
and so the best rank $k$ approximation to $\L_\tG$ is given by 
\begin{equation*}
\L_k \equiv (\L_\tG^+)_k = \sum_{r=1}^{k} \frac{1}{\mu_r} \bnu_r\bnu_r^\tp.
\end{equation*}
The approximation for $\L_G$ is thus given by $\tL=\bC\L_k\bC^\tp=\bC\L_k^{1/2}\L_k^{\tp/2}\bC^\tp = (\L_k^{\tp/2}\bC^\tp)^\tp \L_k^{\tp/2}\bC^\tp$. That is, we can view $\tL$ as the gram matrix of the vectors given by the columns of $\tSv =  (\L_k^{\tp/2}\bC^\tp)$. 

Let us examine $\tSv^\tp=\C\L_k^{\tp/2}$. First consider $\rank(\bC)$, which we claim is $|I|$. Suppose $\bC\f=\zero$, where $\f:I \to \R$. Extend $\f$ to $\hf:[n]\to\R$ by setting $\hf(u)=0$ for all $u\in [n]\setminus I$. Then 
\[(\L_G\hf)(k) = \sum_{i\in[n]}\L_G(k,i)\hf(i) = \sum_{i\in I}\L_G(k,i) \f(i) + \sum_{i\in[n]\setminus I} \L_G(k,i)\hf(i) = \sum_{i\in I}\bC(k,i) \f(i) = 0,\]
implying that $\L_G\hf=\zero$, so $\hf\in\spn(\one))$. However, as long as $|I|\neq[n]$, this is impossible since  $\hf([n]\setminus I)=\zero$. Therefore, so long as $c<n$, we have $\rank(\bC) = c$. 
We now claim that $\rank(\bC\L_k^{\tp/2}) = \rank(\L_k^{\tp/2})$, which is easier to prove in the abstract. 

\begin{lemma}
	Let $\bS:\R^n\to\R^\ell$, $T\in\R^m\to\R^n$ be linear maps with $\rank(\bS)=\ell$. Then $\rank(\bS\T) = \rank(\T)$. 
\end{lemma}
\begin{proof}
	If $\T\f=\zero$ then clearly $\bS\T\f=\zero$ so $\dim\ker(\T)\leq \dim\ker(\bS\T)$. On the other hand, if $\bS\T\f=\zero$ then because $\bS$ is full rank, $\T\f=\zero$ implying that $\dim \ker \T\ge \ker\bS\T$. 	By the rank nullity Theorem (e.g., ~\cite{axler1997linear}) $\rank(\bS\T) + \dim\ker \bS\T = n = \rank(\T) + \dim\ker\T$ from which the result follows immediately. 
\end{proof}

Taking $\bC=\bS$ and $\T=\L_k^{\tp/2}$ in the above lemma gives that $\rank(\bC\L_k^{\tp/2})=k$. Consequently, the vertex matrix $\tSv\in\R^{|I|\times n}$ contains $n$ vectors in $\R^{|I|}$. Moreover, 
\[\rank(\tL) = \rank(\tSv^\tp \tSv) = \rank(\tSv)=k,\]
meaning the $n$ vectors span a $k$-dimensional space. 

One might hope that the approximation matrix $\tL$ was a Laplacian, but this does not seem to be the case in general. While it is true that $\tL(i,i)\geq 0$ (by virtue of being a gram matrix) and that $\tL\one=\zero$ (this follows since $\bC^\tp\one =\zero$ because the rows of $\bC^\tp$ are columns and hence rows of $\L_G$). However, 
\[\tL(i,j) = \sum_{r,s=1}^c \bC(i,r)\bC(j,s)\L_k(r,s),\]
which does not look to be necessarily non-positive. 





