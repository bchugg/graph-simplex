\chapter*{Abstract}
\addcontentsline{toc}{section}{Abstract}



We present and study the graph-simplex correspondence---a tool providing a series of relationships between weighted, undirected graphs on $n$ vertices and simplices in $(n-1)$-dimensional Euclidean space. The core of the correspondence is a  bijection between graphs and hyperacute simplices, first uncovered by Miroslav Fiedler. We consolidate Fiedler's  work on  the subject and expand on it  in several  ways. 


The first relates purely to  the mathematical  properties  of the  correspondence. Among  other things, we extend the correspondence to the normalized Laplacian matrix $\Ln_G$ (whereas previously only the combinatorial  Laplacian,  $\L_G$, had been used), develop new equations and  inequalities relating aspects  of the simplex to those of the graph, and give an isometry between a graph's ``inverse combinatorial  simplex''  and an $n$-dimensional polytope arising from the Laplacian's pseudoinverse. 
%Here, the goal is to convince  the reader that the graph-simplex  correspondence can provide  new insights into the  structure of  both graphs and simplices and is consequently  worthy of being  included in their mathematical  toolbox.  

Secondly, we examine the algorithmic underpinnings and consequences of the correspondence. We begin  by  demonstrating that it can be used to draw  conclusions about the computational complexity of various geometric problems. We then provide lower bounds on the complexity of transitioning between graphs and  simplices, and end by studying  
 low dimensional representations of the simplices.  This provides theoretical justification for recent empirical  work on Laplacian eigenmaps. 

Of possible independent interest, we provide a formula for  the non-zero eigenvalues of $\Ln_G$  in terms of the total weight of spanning trees in the graph $G$, relate the volume of an arbitrary simplex to the eigenvalues of the Gram matrix of its dual simplex (an object we introduce), and give an equation for the adjugate of $\Ln_G$ in terms of the weights of the vertices in  $G$.  
 


\vspace{1cm}
\noindent \textbf{Keywords:}  Graph theory,  simplex geometry, Laplacian matrix, effective resistance,  convex  polyhedra. 




