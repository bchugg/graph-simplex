\chapter*{Abstract}
\addcontentsline{toc}{section}{Abstract}

%The connection between graphs and simplices---a special class of polytope---has remained understudied since  its discovery in 1990s. 
Graphs are ubiquitous in many application  domains, especially in the current age of ``big data''. 
Simplices are fundamental geometric objects, but remain elusive due to their high dimensionality. Among the tools used  to  reason about graphs are various combinatorial techniques, spectral theory, and the probabilistic  method. Simplices,  due to their inherent geometric nature, are mostly studied  by appealing to  geometry. The motivation behind the present work is to introduce another tool with which  we can study graphs and simplices: the \emph{graph-simplex correspondence}. This correspondence provides a relationship between  connected, weighted graphs and  various simplices; at its core is a bijection between graphs and hyperacute simplices.  

We begin by consolidating and elucidating the work of Miroslav Fiedler who  initially discovered the connection. We then expand on the correspondence  in several  ways. The first is purely mathematical. We extend the correspondence to the normalized Laplacian matrix, develop new equations and  inequalities relating the aspects  of the simplex to those of the graph, and discuss the correspondence as it pertains to random walks. Here, the goal is to convince  the reader that the graph-simplex  correspondence is worthy of being  included in their mathematical  toolbox.  Secondly, we examine the algorithmic underpinnings of the correspondence. We explore how quickly various aspects of  the correspondence can be  computed---computing the simplex of a graph or the graph of a simplex, for example. After giving lower bounds on such questions, we turn to various  approximations.  

\vspace{1cm}
\noindent \textbf{Keywords:}  Graph theory,  simplex geometry, high-dimensional  geometry, effective resistance,  convex  polyhedra. 




