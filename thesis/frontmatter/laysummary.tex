\chapter*{Lay Summary}
\addcontentsline{toc}{section}{Lay Summary}

The  most  significant features of mathematical research, to the astonishment of  many, do not involve generating contrived  calculus  questions with which to torture sleep-deprived undergraduates. 
Instead, one central focus of research is  on further developing  its different branches---geometry, probability, number theory, etc.  Another concern, however, is to seek connections between these different areas. Such  connections are elusive, but often point to some deeper and beautiful (stay with me) mathematical structure. For example, 

This dissertation is concerned with research of the latter type. In  the 1990s, Miroslav Fiedler began exploring what we are calling the ``graph-simplex correspondence''. 
The  reader is invited  to draw several dots  on a piece of paper and connect each one with  several (or all) of the others by drawing  lines between them. There, you have  just succeeded in drawing a graph. Your graph can be  described  by listing the dots (formally  called vertices), and whether or not there is a  connection between  them. Regardless of how far  apart the dots  are on the page are, we are  simply interested in whether or not there is a  connection between two vertices. Thus, a graph  lacks inherent  geometry; it  can  be described simply with finite  lists. A simplex, on the other  hand, is essentially a triangle but generalized  to higher dimensions. Is it therefore inherently geometric, which makes a connection between graphs and simplices all the more surprising. 

When studying the abstract, one can never be  sure where whether one's work will remain only of interest to mathematicians or will find some application. That being said, we expect this research to be highly applicable---it will most likely help develop interstellar travel, clarify broad macroeconomic trends, and 6G. Just kidding. We do hope, however, that this work will serve to inspire researchers to include the  graph-simplex correspondence as a tool to use when  investigating graphs and/or simplices, and will thereby contribute to future  research. 
 

