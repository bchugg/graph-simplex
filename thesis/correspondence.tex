\chapter{The Graph-Simplex Correspondence}



\section{Convex Polyhedra of Matrices}
\note{Unclear whether to leave this or just discuss the simplex of a graph immediately.}
Consider an arbitrary real and symmetric matrix $\M\in \R^{n\times n}$ which admits the eigendecomposition $\M=\sum_{i=1}^d \lambda_i \eig_i\eig_i^\tp$ for some $d\leq n$ (i.e., $\M$ has eigenvalue zero with multiplicity $n-d$) where the eigenvectors $\{\eig_i\}_{i=1}^d $ are orthonormal. Writing out the eigendecomposition as 
\[\M=\Eig_M\Eval_M\Eig_M^\tp=(\Eig_M\Eval_M^{1/2})(\Eig_M\Eval_M^{1/2})^\tp,\]
with $\Eig_M = (\vp_1,\dots,\vp_d)$, $\Eval_M=\diag(\lambda_1,\dots,\lambda_d)$ (note the respective absences of $\vp_{d+1},\dots,\vp_n$ and $\lambda_{d+1},\dots,\lambda_n$), suggests that we might consider $\Eval_M^{1/2}\Eig_M$ as a vertex matrix, thus $\M$ as a gram matrix. 
Inorexably compelled by this intuition, define the vertices $\sv_1,\dots,\sv_n$ given by 
\begin{equation*}
    \sv_i = (\Eval_\M^{1/2}\Eig_\M)^\tp (\cdot,i) = (\eig_i(1)\lambda_1^{1/2}, \eig_i(2)\lambda_2^{1/2},\dots,\eig_i(n)\lambda_n^{1/2})^\tp,
\end{equation*}
and the polytope 
\begin{equation*}
\pol_\M=\conv(\sv_1,\dots,\sv_n).
\end{equation*}
\note{define convex hull in prelims.}
We call $\pol_\M$ the \emph{the polytope of the matrix $\M$}. 
Letting $\Sv(\pol_\M)=(\sv_1,\dots,\sv_n)\in \R^{d\times n}$ be the matrix whose $i$-th column is the $i$-th vertex $\sv_i$, we see that 
$ \Sv=(\Eig \Eval^{1/2})^\tp$, and 
\begin{equation*}
    \Sv^\tp \Sv=(\Eig\Eval^{1/2}) (\Eig\Eval^{1/2})^\tp = \Eig \Eval\Eig^\tp =\M.
\end{equation*}
Observe that the polytope $\splx(\M)$ is $d$-dimensional: \note{Introduce dimension of polytope in prelims}
\[\rank(\Sv)=\rank(\Sv^\tp \Sv)=\rank(\M)=d.\]


\paragraph{The Inverse Polytope.}
With $\M$ as above, consider the pseudo-inverse of $\M$ which we can write as 
\[\M^+=\sum_{i=1}^d \lambda_i^{-1} \eig_i\eig_i^\tp = \Eig_\M \Eval_\M^{-1/2}\Eig_M.\]


We can thus associated with $\M^+$ a polytope $\pol_{\M^+}$, which has as its vertex matrix $\Sv(\pol_{\M^+})=(\Eig\Eval^{-1/2})^\tp$; that is, the vertices $\{\sv_i^+\}$ of $\pol_{\M^+}$ are defined by 
$\sv_i^+(j)=\eig_j(i)/\lambda_j^{1/2}$. We call $\pol_{\M^+}$ the \emph{inverse polytope of $\M$}. 

\subsection{The Simplex of a Graph}
\label{sec:graph_to_simplex}
For an undirected graph $G$, the previous section yields several polytopes corresponding to $G$. The most structurally rich among these are the polytopes $\splx_G\equiv \pol_{\L_G}$ and $\splxn_G\equiv \pol_{\Ln_G}$ corresponding to $G$'s combinatorial and normalized Laplacians.  We let $\Sv_G=(\sv_1,\dots,\sv_n)$ and $\Svn_G=(\svn_1,\dots,\svn_n)$ denote the vertices of $\splx_G$ and $\splxn_G$, respectively. Since $\rank(\L_G)=\rank(\Ln_G)=n-1$, the polytopes $\splx_G$ and $\splxn_G$ are in fact simplices. Consequently, we will often refer to $\splx_G$ as the \emph{simplex of $G$}, and to $\splxn_G$ as the \emph{normalized simplex of $G$}. If $G$ is clear from context we will often drop it from the subscript. 

\note{Why do we need this? We know the simplices are well-defined  due  to the rank}
\begin{lemma}
\label{lem:sv_affine_indep}
The vertices $\{\sv_i\}$  and $\{\svn_i\}$ are affinely independent. 
\end{lemma}
\begin{proof}
Suppose $\balpha = (\alpha_1,\dots,\alpha_n)$ is such that 
$\sum_{i=1}^n \alpha_i\sv_i = \zero$, i.e., $\balpha\in\ker(\Sv)$. Since $\ker(\Sv)=\ker(\Sv^\tp\Sv)=\ker(\L)=\spn(\{\one\})$, there exists some $k\in \R$ such that $\balpha=k\one$. If $\la \balpha,\one\ra = \la k\one,\one\ra=kn=0$ however, then we must have $k=0$, demonstrating that $\alpha_i=0$ for all $i$. Hence the vectors $\{\sv_i\}$ are affinely independent. Likewise, if $\balpha\in\ker(\Svn)=\ker(\Ln)=\spn(\{\sqrt{\w}\})$, then $\balpha=k\sqrt{\w}$. But $\la k\sqrt{\w},\one\ra = k\sum_{i}w(i)=0$, so $\balpha=\zero$. 
\end{proof}

For the inverse simplex and normalized simplex of $G$ we have 
\[\Sv^+ = \Eval^{-1/2}\Eig^\tp,\qquad \text{and} \qquad \Svn^+ = \Evaln^{-1/2}\Eign^\tp.\]
Let $\widetilde{\Eig}$ be the matrix containing all eigenvectors of $\L_G$ (i.e., also containing $\one/\sqrt{n}$).  Note that because $\wEig^\tp \wEig=\I$ it follows that $\wEig\wEig^\tp=\I$ as well. Therefore, 
\begin{equation*}\delta_{i,j}=\sum_{k=1}^n \vp_k(i)\vp_k(j) = \sum_{k=1}^{n-1} \vp_k(i)\vp_k(j) + 1/n.
\end{equation*}
From this, it follows that 
\[\la\sigma_i^+,\sigma_j\ra = \delta_{i,j} - 1/n,\]
hence, 
\begin{equation}
\label{eq:sv+sv}
    \Sv^\tp\Sv^+=(\Sv^+)^\tp\Sv=\I-\one\one^\tp/n.
\end{equation}
For the inverse normalized simplex, on the other hand, one has $\vpn_{n}(i)\vpn_{n}(j)=\sqrt{w(i)w(j)}/n$, implying that 
\begin{equation*}\delta_{i,j}=\sum_{k=1}^n \vpn_k(i)\vpn_k(j) = \sum_{k=1}^{n-1} \vpn_k(i)\vpn_k(j) + \frac{\sqrt{w(i)w(j)}}{n},
\end{equation*}
and so 
\begin{equation}
\label{eq:svn+svn}
\Svn^\tp\Svn^+=(\Svn^+)^\tp\Svn=\I-\sqrt{\w}\sqrt{\w}^\tp/n,
\end{equation}
where $\w=(w(1),\dots,w(n))$ and $\sqrt{\w}=(\sqrt{w(1)},\dots,\sqrt{w(n)})$. 

Exploring the relationships between the vertex matrices and themselves meanwhile,  we find that 
\begin{align}
\Sv^+(\Sv^+)^\tp &= 
\begin{pmatrix}
\sum_i \sv_i^+(1)\sv_i^+(1) & \dots & \sum_i \sv_i^+(1)\sv_i^+(n) \\
\vdots &\ddots  & \vdots \\
\sum_i \sv_i^+(n)\sv_i^+(1) & \dots & \sum_i \sv_i^+(n)\sv_i^+(n)
\end{pmatrix}\notag 
\\[10pt] 
&= 
\begin{pmatrix}
\lambda_1^{-1} \la \vp_1,\vp_1\ra & \dots & \lambda_1^{-1/2}\lambda_n^{-1/2}\la \vp_1,\vp_n\ra \\
\vdots & \ddots & \cdots \\
\lambda_1^{-1/2}\lambda_n^{-1/2}\la \vp_n,\vp_1\ra & \dots &\lambda_n^{-1} \la \vp_n,\vp_n\ra 
\end{pmatrix} = \Eval^{-1},\label{eq:SvSv+}
\end{align}
and likewise, 
\begin{equation}
\label{eq:SvnSvn+}
\Svn^+(\Svn^+)^\tp = \Evaln^{-1}.
\end{equation}






\subsection{The Graph of a Simplex}
\label{sec:simplex_to_graph}
In this section we demonstrate that each hyperacute (more precisely, each non-obtuse) simplex is the inverse simplex of a graph $G$. 

\begin{lemma}[\cite{fiedler1993geometric}]
Given a simplex $\ssplx\subset\R^{n-1}$ centered at the origin, let $\vb*{Q}$ be the Gram matrix of its normalized outernormals. That is, $\vb*{Q}(i,j)=\la \u_i,\u_j\ra$ where $\u_{i}$ is the outer normal to the face $\ssplx_\ic$. If $\vb*{Q}_1,\vb*{Q}_2\in\R^{n\times n}$ are defined by 
\begin{equation*}
\vb*{Q}_1 = \diag\bigg(\norm{\alt(\splx_1)}_2^{-1},\dots,\norm{\alt(\splx_n)}_2^{-1}\bigg),
\end{equation*}
and 
\begin{equation*}
\vb*{Q}_2(i,j) = \begin{cases}
1,& \text{if } i=j,\\
-\cos\theta_{i,j},&\text{otherwise},
\end{cases} 
\end{equation*}
where $\theta_{i,j}$ is the (interior) angle between $\ssplx_\ic$ and $\ssplx_\jc$, then 
\begin{equation*}
\vb*{Q} = \vb*{Q}_1 \vb*{Q}_2 \vb*{Q}_1.
\end{equation*}
\end{lemma}


Let $\splx^+$ be a hyperacute simplex, and $\splx$ its dual. The vertex matrix $\Sv$ of $\splx$ contains the outer normals of $\splx^+$ (see discussion on dual simplex in Section \ref{sec:dual_simplex}). Hence, taking $\vb*{Q}=\Sv^\tp \Sv$ in the above Lemma applied to the simplex $\splx^+$, we obtain explicit entries for the gram matrix $\Sv^\tp\Sv$: 
\begin{equation*}
    \Sv^\tp \Sv(i,j) = \begin{cases}
    \norm{\alt(\splx_i^+)}_2^{-2},& \text{if }i=j,\\
-\cos\theta^+_{i,j} \norm{\alt_i^+}_2^{-1}\cdot \norm{\alt_j^+}_2^{-1},& \text{if }i\neq j.
    \end{cases}
\end{equation*}
(Here $\theta^+{i,j}$ is the angle between $\splx_\ic^+$ and $\splx_\jc^+$.)
We claim that $\Sv^\tp\Sv$ is the Laplacian matrix of some graph $G$. First, the matrix is symmetric. Second,
for each $i$, $(\Sv^\tp\Sv)(i,i)=\norm{\alt_i^+}_2^{-2}>0$, and for $i\neq j$, $(\Sv^\tp\Sv)(i,j) \leq 0$ since $\theta^+_{i,j}\leq \pi/2$ by assumption (note therefore the importance that $\splx^+$ is hyperacute). Finally, denote $\Sv=(\sv_1,\dots,\sv_n)$, and recall from the construction of the dual simplex in Section \ref{sec:dual_simplex} that $\sv_n=-\sum_{i<n}\sv_i$. Therefore, for $i\neq n$, 
\begin{align*}
    \sum_{j=1}^n (\Sv^+\Sv)(i,j) &= \sum_{j=1}^{n-1} \la \sv_i,\sv_j\ra + \la \sv_i,-\sum_{j<n}\sv_j\ra = \sum_{j<n}\la \sv_i,\sv_j\ra - \sum_{j<n}\la \sv_i,\sv_j\ra  = 0,
\end{align*}
hence $\Sv^\tp\Sv\one=\zero$, meaning that 
\[(\Sv^*\Sv)(i,i) = -\sum_{j\neq i}(\Sv^*\Sv)(i,j).\]
If we construct a weighted graph $G=(V,E,\w)$ on $n$ vertices with edge weights $\w(i,j) = (\Sv^\tp\Sv)(i,j)$, it then follows that $\Sv^\tp\Sv=\L_G$. 

We summarize the material in Sections \ref{sec:graph_to_simplex} and \ref{sec:simplex_to_graph} with the following theorem. 

\begin{theorem}
	\label{thm:graph-simplex}
	There exists a bijection between hyperacute \note{Need  to  define  hyperacute as allowing angles of $\pi/2$} simplices in $\R^{n-1}$ centered at the origin and connected, weighted graphs on $n$ vertices. 
\end{theorem}

\note{in light of the theorem, may  want to think about structuring the discussion so that $\splx^+$ is actually the main object, while $\splx$ is secondary. }

\section{Simplices of Special Graphs}

\paragraph{Simplex of Complement Graph, $G^c$}
Suppose $G$ is unweighted. The complement of $G$, $G^c$, has adjacency matrix $\A_{G^c}=\one\one^\tp - \I-\A_G$ and degree matrix $\D^c=\D_{G^c}=(n-1)\I-\D_G$ since $\deg(i)_{G^c}=n-1-\deg(i)_G$. The Laplacian of $G^c$ thus reads as 
\[\L^c=\D^c-\A^c=n\I-\D_G-\one\one^\tp+\A_G=n\I-\one\one^\tp-\L_G.\]
Of course, $\one$ is still an eigenfunction of $\L^c$. For $\vp\perp \one$, we have 
\[\L^c\vp = n\vp - \one\la \one,\vp\ra - \L\vp = (n-\lambda)\vp,\]
which which it follows that $\L^c$ shares the same eigenfunctions as $\L$, with corresponding eigenvalues $\{n-\lambda_i\}$. Consequently, the simplex corresponding to $G^c$, $\splx^c$ has vertices given by 
\[\sigma_i(j) = \vp_j(i) \sqrt{n-\lambda_j},\]
and the inverse simplex has vertices 
\[\sigma_i(j)^+=\frac{\vp_j(i)}{\sqrt{n-\lambda_j}}. \]



\paragraph{Subgraphs}
Let $H\subset G$, in the sense that $w_H(i,j)\leq w_G(i,j)$ for all $i,j\in [n]$. Then, for any $f:V\to\R$ we see that 
\[\Lf_G(f)=\sum_{i\sim j}w_G(i,j)(f(i)-f(j))^2 \geq \sum_{i\sim j}w_H(i,j)(f(i)-f(j))^2 =\Lf_H(f).\]
Therefore, 
\begin{equation*}
\norm{\Sv_H f}_2^2 \leq \norm{\Sv_G f}_2^2. 
\end{equation*}
In particular, taking $f=\chi_i$ for any $i$, this yields $\norm{\sv_i(G)}_2^2\geq \norm{\sv_i(H)}_2^2$. 

If $G$ is a multiple of $H$ such that $w_G(i,j)=c\cdot w_H(i,j)$ for all $i,j$, then we see that $\Lf_G(f)=c\cdot \Lf_H(f)$ so that $\norm{\sigma_i(G)}_2^2 =c\cdot \norm{\sv_i(H)}_2^2$. Meanwhile however, the normalized simplex is unaffected by the re-weighting: 
\begin{align*}\Lnf_G(f)&=\sum_{i\sim j}w_G(i,j)\bigg(\frac{f(i)}{\sqrt{w_G(i)}}-\frac{f(j)}{\sqrt{w_G(j)}}\bigg)^2\\
&= \sum_{i\sim j}c\cdot w_H(i,j)\bigg(\frac{f(i)}{\sqrt{c\cdot w_H(i)}}-\frac{f(j)}{\sqrt{c\cdot w_H(j)}}\bigg)^2 \\
&= \sum_{i\sim j} w_H(i,j)\bigg(\frac{f(i)}{\sqrt{ w_H(i)}}-\frac{f(j)}{\sqrt{ w_H(j)}}\bigg)^2 = \Lnf_H(f).
\end{align*}

\paragraph{Product Graphs}

\begin{definition}
	\label{def:product_graphs}
	Given two graphs $G=(V(G),E(G))$ and $H=(V(H), E(H))$, the \emph{product graph of $G$ and $H$} is the graph with vertex set $V(G)\times V(H)$ and edge set
	$\{((i_1,j),(i_2,j)):(i_1,i_2)\in E(G), j\in V(H)\}\cup\{((i,j_1),(i,j_2)):(j_1,j_2)\in E(H), i\in V(G)\}$. It is typically denoted $G\times H$. 
\end{definition} 

Suppose $G$ has eigenvalues $\lambda_1\geq \dots\geq \lambda_{n}$ and corresponding eigenvectors $\vp_1,\dots,\vp_{n}$ as usual. Let $H$ have eigenvalues $\mu_1\geq \dots\geq \mu_{m}$ and corresponding eigenvectors $\bpsi_1,\dots,\bpsi_{m}$. 
We claim that $G\times H$ has $m+n$ eigenvalues $\{\lambda_i+\mu_j\}_{i\in[n],j\in[m]}$ with eigenvectors $\{f_{i,j}\}_{(i,j)\in[n]\times[m]}$ given by 
\[f_{i,j}(k,\ell) = \vp_i(k)\bpsi_j(\ell).\]
Indeed: 
\begin{align*}
(\L_{G\times H} f_{uv})(ij) &= \deg_{G\times H}((i,j))f_{uv}(ij) - \sum_{(k,\ell)\in \delta((i,j))} f_{uv}(k\ell) \\
&= (\deg_G(i)+\deg_H(j))\vp_u(i)\bpsi_v(j) - \sum_{(k,\ell)\in \delta_{G\times H}((i,j))} \vp_u(i)\bpsi_v(j) \\
&= (\deg_G(i)+\deg_H(j))\vp_u(i)\bpsi_v(j) - \sum_{k\in \delta_G(i)} \vp_u(k)\bpsi_v(j) - \sum_{\ell\in \delta_H(j)} \vp_u(i)\bpsi_v(\ell) \\
&=\bigg(\deg_G(i)\vp_u(i) - \sum_{k\in\delta_G(i)}\vp_u(k)\bigg)\bpsi(j) + \bigg(\deg_H(j)\bpsi_v(j)-\sum_{\ell\in\delta_H(j)}\bpsi_v(\ell)\bigg)\vp_u(i) \\
&= (\L_G \vp_u)(i) \cdot \bpsi_v(j) + (\L_H\bpsi_v)(j)\cdot \vp_u(i) \\
&= \lambda_u\vp_u(i) \bpsi_v(j) + \mu_v\bpsi_v(j)\vp_u(i) \\
&= (\lambda_u+\mu_v)\vp_u(i)\bpsi_v(j) = (\lambda_u+\mu_v)f_{uv}(ij),
\end{align*}
as desired. Consequently, the product graph yields a simplex with vertices 
\[\sv_{ij}(k\ell)=f_{k\ell}(ij)(\lambda_k+\mu_\ell)^{1/2}.\]


\subsection{Examples}

\note{Explore certain graphs whose eigenvalues and eigenvectors are easy to compute.}

\paragraph{The Complete Graph, $K_n$.}
First let us consider the combinatorial simplex, $\splx^\comb(K_n)$.  The combinatorial Laplacian $\L_{K_n}$ has two eigenvalues: 0 with multiplicity 1 and $n$ with multiplicity $n-1$. To see this, observe that for any $\eig$ perpendicular to $\one$, we have 
\begin{align*}
\L_{K_n}\eig &= \bigg(\vp(1)(n-1)-\sum_{i\neq 1}\eg(i),\dots,\vp(n)(n-1)-\sum_{i\neq n}\eg(i)\bigg) \\
&= \bigg(\vp(1)n - \sum_i \vp(i), \dots,\vp(n)n - \sum_i \vp(i)\bigg) \\
&= (\vp(1)n,\dots,\vp(n)n) = n\eig,
\end{align*}
since $\sum_i\vp(i)=\la \eig,\one\ra =0$. \note{finish this}

\paragraph{Cycle Graph}

\paragraph{Path Graph}


\section{Properties of \texorpdfstring{$\splx_G$}{Graph Simplices}}
Fix a graph $G=(V,E,w)$ with $|V|=n$. The first property of $\splx=\splx_G$ that's worth noting is that it is centred at the origin: $\cent(\splx) = n^{-1}\Eval^{-1/2}\Eig^\tp \one =\zero$, since $\la \vp_i,\one\ra =0$ for all $i<n$.  Likewise, $\cent(\splx^+)=\zero$. 




\note{Should hold between every  simplex and its dual. Probably move into prelims section}
\begin{lemma}
\label{lem:faces_orthogonal}
Let $\splx$ and $\splx^+$ be the simplex and inverse simplex of a graph $G=(V,E)$. For any non-empty $U\subset V$,  
the faces $\splx^+_U$ and $\splx_{U^c}$ are orthogonal. In other words, if $\p_1,\p_2\in \splx_U^+$ and $\q_1,\q_2\in \splx_{U^c}$, then $\la \p_1-\p_2,\q_1-\q_2\ra=0$. 
\end{lemma}
\begin{proof}
Let $\p\in \splx^+_U$ and $\q\in \splx_{U^c}$. Letting their barycentric coordinates be $\x_{\p}$ and $\x_{\q}$ respectively, write 
\[\la \p,\q\ra = \x_{\p} (\Sv^+)^\tp \Sv \x_{\q}=\x_{\p}(\I-\one\one^\tp/n) \x_{\q},\]
where we've employed Equation \eqref{eq:sv+sv}. Now, $\x_{\p}(i)=0$ for all $i\in U^c$ and $\x_{\q}(j)=0$ for all $j\in U$. Therefore, $\la \x_{\p},\x_{\q}\ra =0$. Moreover, $\norm{\x_{\p}}_1=\norm{\x_{\q}}_1=1$ and so the above simplifies to $\la \p,\q\ra = -1/n$. Consequently, if $\p_1,\p_2\in \splx_U^+$ and $\q_1,\q_2\in\splx_{U^c}$ we have 
\[\la \p_1-\p_2,\q_1-\q_2\ra = 0,\]
completing the proof. 
\end{proof}

The following lemma presents an alternate characterization of the simplex. 

\begin{lemma}
\label{lem:S_alt_desc}
For a simplex $\splx$ of a graph $G$, 
\begin{equation}
\label{eq:S_alt_desc}
    \splx=\bigg\{\x\in \R^{n-1}:\x^\tp\Sv^++\frac{\one^\tp}{n}\geq \zero^\tp\bigg\}.
\end{equation}
\end{lemma}
\begin{proof}
Put $E=\{\x\in \R^{n-1}:\x^\tp\Sv^++\one^\tp/n\geq \zero^\tp\}$. First we show that $E\subset\splx$. 
Since $\rank(\Sv)=n-1$, it follows that given any $\x\in E$ (indeed, any $\x\in\R^{n-1}$) we can write $\x=\Sv\y$ for some $\y\in\R^n$. Letting $\bar{y}=n^{-1}\sum_i y(i)$ be the mean of the vector $\y$, compute
\begin{align*}
    \x^\tp\Sv^+ &= \y^\tp \Sv^\tp\Sv^+ = \y^\tp(\I-\one\one^\tp/n)=\y^\tp - \bar{y}\one^\tp.
\end{align*}
If $\x\in E$ the above implies that 
\[\y^\tp-\bar{y}\one^\tp + \one^\tp/n\geq \zero^\tp.\]
Moreover, since $\Sv\one=\zero$, we have $\x=\Sv\y=\Sv(\y-\bar{y}\one +\one/n)$. Noticing that 
\[\la \y-\bar{y}\one+\one^\tp/n,\one\ra = n\bar{y}-n\bar{y}+1=1,\]
demonstrates that the vector $\widetilde{\y}=\y-\bar{y}\one+\one^\tp/n$ is a barycentric coordinate for $\x$, and so $\x\in \splx$. 

Conversely, for $\x\in\splx$ let $\y$ be its barycentric coordinate. Then 
\[\x^\tp\Sv^++\one^\tp/n=\y^\tp(\I-\one\one^\tp/n)+\one^\tp/n=\y^\tp - \one^\tp/n+\one^\tp/n=\y^\tp\geq \zero^\tp, \]
hence $\splx\subset E$. This completes the proof. 
\end{proof}

\begin{lemma}
\label{lem:SUsubset}
Let $\splx$ be the simplex of a graph $G=(V,E,w)$, and fix $U\subset V$. For any non-empty $E\subset U^c$,
\begin{equation*}
    \splx_U \subset \bigg\{\x\in\R^{n-1}: \sum_{i\in E}\la \x,\sv_i^+\ra +\frac{|E|}{n}=  0\bigg\}.
\end{equation*}
\end{lemma}
\begin{proof}
Let $\x\in\splx_U$ be arbitrary. For any $i\in U^c$ we have $\la \p,\sv_i^+\ra = -1/n$. Hence, for any $E\subset U^c$
\[\sum_{i\in E}\la \x,\sv_i^+\ra + \frac{|E|}{n} = \sum_{i\in E}\bigg(\la \x,\sv_i^+\ra+\frac{1}{n}\bigg)=\sum_{i\in E}\bigg(\frac{1}{n}-\frac{1}{n}\bigg)=0,\]
implying that $\x$ is in the desired set. 
\end{proof}

Lemma \ref{lem:SUsubset} gives us an alternate way to prove Lemma \ref{lem:S_alt_desc}. For any $i$,  taking $U=N\setminus \{i\}$ and $E=\{i\}$, it implies that $\splx_{\{i\}^c}$ is a subset of the hyperplane 
\[ \H_i\equiv \{\x\in\R^{n-1}:\la \x,\sv_i^+\ra+1/n=0\}.\]
All points in the simplex $\splx$ lie to one side of $\splx_{\{i\}^c}$, i.e., they lie in the halfspace 
\[\H_i^{\geq } \equiv \{\x\in\R^{n-1}:\la \x,\sv_i^+\ra+1/n\geq0\}.\]
(We know it is this halfspace because $\zero\in\splx\cap\H_i^{\geq }$.) The simplex is the interior of the region defined by the intersection of the faces $\splx_{\{i\}^c}$, i.e.,   
\begin{equation*}
    \splx=\bigcap_i \H_i.
\end{equation*}
Moreover, $\x\in \bigcap_i\H_i$ iff $\la \x,\sv_i^+\ra +1/n\geq 0$ for all $i$, i.e., $(\la \x,\sv_1^+\ra,\dots,\la \x,\sv_n^+\ra) + \one/n\geq \zero$, meaning $\x$ satisfies \eqref{eq:S_alt_desc}.  






\paragraph{Global Connectivity}
Given $U\subset V(G)$ then \emph{cut-set} of $U$ is 
\[\delta U \equiv (U\times U^c)\cap E(G)= \{(i,j)\in E(G): i\in U, j\in U^c\}.\]
Noting that $|\chi_U(i)-\chi_U(j)|=\chi_{(i,j)\in \delta U}$, we see that
\begin{align*}
    w(\delta U) = \sum_{i,j\in E} w(i,j)|\chi_U(i)-\chi_U(j)| = \sum_{i,j\in E} w(i,j)(\chi_U(i)-\chi_U(j))^2 = \Lf(\chi_U). 
\end{align*}
Moreover, $\norm{c(\splx_U)}^2_2 = \la |U|^{-1} \Sigma\chi_U,|U|^{-1} \Sigma\chi_U\ra = |U|^{-2} \Lf(\chi_U)$ and so 
\begin{equation}
\label{eq:c(S_U)}
    \norm{c(\splx_U)}_2^2 = \frac{w(\delta U)}{|U|^2}.
\end{equation}
Via the same process we can also obtain an equivalent expression for the centroid of the inverse simplex: 
\begin{equation}
\label{eq:c(S+_U)}
    \norm{c(\splx^+_U)}_2^2 = \frac{w(\delta^+ U)}{|U|^2},
\end{equation}
where we define $w(\delta^+U) \equiv  \la \Sv^+\chi_U,\Sv^+\chi_U\ra=\la \chi_U,\L^+\chi_U\ra$. 




\subsubsection{Centroid and Altitudes}
Recall that the altitude between $\splx[U]$ and $\splx[U^c]$ of a simplex $\splx$ is the unique vector $\p-\q$ where  $\p\in \splx_{U^c}$ and $\q\in\splx_{U}$ which lies in the orthogonal complement of both $\splx_U$ and $\splx_{U^c}$. 

\begin{lemma}
\label{lem:complimentary_centroids}
Let $U\subset V$ be non-empty. Then the vectors $c(\splx_U)$ and $c(\splx_{U^c})$ are antiparallel. In particular, $(n-|U|)c_{U^c} = |U|c_{U}$ and 
\[\frac{c_U}{\norm{c_U}_2}=-\frac{c_{U^c}}{\norm{c_{U^c}}_2}.\]
\end{lemma}
\begin{proof}
This is a straightforward computation: Observing that $\chi_U=n-\chi_{U^c}$ we have  
\[c_U = |U|^{-1}\Sv\chi_U = |U|^{-1}\Sv(\one-\chi_{U^c})=-|U|^{-1}\Sv\chi_{U^c}=-|U|^{-1}\frac{|U^c|}{|U^c|}\Sv\chi_{U^c}=\frac{n-|U|}{|U|}c_{U^c},\]
where we've used that $\Sv\one=\zero$. This proves the first result; the second follows from normalizing the two vectors.
\end{proof}

\begin{lemma}
\label{lem:alt_and_centroid}
For a simplex $\splx$ of a graph $G=(V,E)$ and any $U\subset V$, $U\neq\emptyset$, 
\begin{equation*}
    \frac{\alt(\splx_U)}{\norm{\alt(\splx_{U})}_2}=\frac{c^+(\splx_{U^c})}{\norm{c^+(\splx_{U^c})}_2}=-\frac{c^+(\splx_{U})}{\norm{c^+(\splx_{U})}_2},
\end{equation*}
and 
\begin{equation*}
    \frac{\alt^+(\splx_U)}{\norm{\alt^+(\splx_{U})}_2}=\frac{c(\splx_{U^c})}{\norm{c(\splx_{U^c})}_2}=-\frac{c(\splx_{U})}{\norm{c(\splx_{U})}_2}.
\end{equation*}
\end{lemma}
\begin{proof}
We prove the first set of  equalities only; the second is obtained similarly. 
Put $\alt_U=\alt(\splx_U)$ and $c_U=c(\splx_U)$. By definition, $\alt_U$ is orthogonal to both $\splx_U$ and $\splx_{U^c}$. \note{need the following claim: Any vector perpendicular to $\splx_U$ can be written as $\Sv^+x_{U^c}$. Why the hell is this true? $\splx^+x_{U^c}$ represents the simplex $\splx^+_{U^c}$ which we know is perpendicular to $\splx_U$. However, does it follow it is the \emph{only} thing perpendicular to $\splx_U$?? } Since $\alt_U$ begins at $\splx_U$ and ends at $\splx_{U^c}$ it follows that 
\begin{equation*}
    \frac{\alt_U}{\norm{\alt_U}_2}=-\frac{\Sv^+f_{U^c}}{\norm{\Sv^+f_{U^c}}_2}=\frac{\Sv^+f_{U}}{\norm{\Sv^+f_{U}}_2}.
\end{equation*}
By Lemma \ref{lem:complimentary_centroids}, taking $f_{U^c}=\chi_{U^c}/|U^c|$ and $f_U=\chi_U/|U|$ yields a solution to the above equation. We claim there are no other solutions, up to scaling. Indeed, let $g_{U^c}$ and $g_U$ satisfy the above, and normalize them so that $\norm{\Sv^+g_{U^c}}_2=\norm{\Sv^+g_U}_2=1$. Then we have $\Sv^+(g_U+g_{U^c})=0$ and so $g_U+g_{U^c}=k\one$ since $\ker(\Sv^+)=\spn(\{\one\})$. Hence $g_U$ and $g_{U^c}$ are scaled versions of $f_U$ and $f_{U^c}$.  
\end{proof}

\begin{lemma}
\label{lem:||alt||}
For any non-empty $U\subset V$, $\norm{a^+_U}_2^2=1/w(\delta U)$ and $\norm{a_U}_2^2 = 1/w(\delta^+ U)$.  
\end{lemma}
\begin{proof}
By definition of the altitude there exists barycentric coordinates $\x_U$ and $\x_{U^c}$ such that $a^+U = \Sv^+(\x_U-\x_{U^c})$. Combining this representation of $a^+_U$ with that given by Lemma \ref{lem:alt_and_centroid}, write 
\begin{align*}
    \norm{a^+_U}_2 = \frac{\la a^+_U,a^+_U\ra }{\norm{a^+_U}_2} = \frac{\la \Sv^+(\x_{U^c}-\x_{U}), c_{U^c}\ra }{\norm{c_{U^c}}_2} = \frac{\la \Sv^+(\x_{U^c}-\x_{U}), \Sv\bchi_{U^c}\ra }{\sqrt{w(\delta U^c)}},
\end{align*}
where the final equality comes from using the definition of the centroid in the numerator, and Equation \ref{eq:c(S_U)} in the denominator. Recalling the relation between $\Sv$ and $\Sv^+$ given by Equation \ref{eq:sv+sv} and that $\x_U$ and $\x_{U^c}$ are barycentric coordinates, we can rewrite the above as 
\[\frac{(\x_{U^c}-\x_{U})^\tp(\I-\one\one^\tp/n)\bchi_{U^c}}{\sqrt{w(\delta U^c)}}=\frac{1}{\sqrt{w(\delta U^c)}}. \]
Squaring both sides while noting that $\delta U = \delta U^c$ completes the proof of the first equality. For the second, we proceed in precisely the same manner to obtain $\norm{a_U}_2^2 = 1/w(\delta^+ U^c)$. However, it's not immediately obvious that $w(\delta^+U^c)=w(\delta^+U)$. To see this, first recall that $\Sv^+\one = \Eval^{-1/2}\Eig^\tp\one = \zero$, and so 
\begin{align*}
w(\delta^+ U^c) &= \la \Sv^+\bchi_{U^c},\Sv^+\bchi_{U^c}\ra  \\
&= \la \Sv^+(\one - \bchi_{U}),\Sv^+(\one - \bchi_{U})\ra  \\
&= \la \Sv^+\bchi_{U},\Sv^+\bchi_{U}\ra = w(\delta^+ U).
\qedhere
\end{align*}
\end{proof}


\begin{lemma}
\label{lem:svi^+ai_parallel}
The vectors $\sv_i^+$ and $\alt(\splx_i)$ are antiparallel. 
\end{lemma}
\begin{proof}
First, notice that $\sv_i^+=\bchi_i \Sv^+=\cent(\splx^+_{\{i\}})$ and so \begin{equation*}
    \norm{\sv_i^+}_2 = \norm{\cent(\splx^+_{\{i\}})}_2 = \norm{ w(\delta^+ \{i\})}_2 = \frac{1}{\norm{\alt_i}}_2,
\end{equation*}
where the penultimate and final inequalities follow from Equation \eqref{eq:c(S+_U)} and Lemma \ref{lem:||alt||}, respectively. Let $\x=\x_{\{i\}^c}$ be the barycentric coordinate of the face $\S_{\{i\}^c}$ such that $\alt_i = \Sv(\x-\bchi_i)$. Then, 
\begin{align*}
    \bigg\la \frac{\sv_i^+}{\norm{\sv_i}_2},\frac{\alt_i}{\norm{\alt_i}_2}\bigg \ra 
    &= \frac{1}{\norm{\sv^+_i}_2\norm{\alt_i}_2} \bigg( \la \sv_i^+,\Sv\x\ra - \la \sv_i^+,\Sv\bchi_i\ra \bigg) \\
    &= \bchi_i^\tp (\Sv^+)^\tp \Sv \x - \bchi_i^\tp (\Sv^+)^\tp \Sv\bchi \\
    &= \bchi_i^\tp (\I-\J/n) \x - \bchi_i^\tp(\I-\J/n) \bchi_i \\
    &= - \frac{\bchi_i^\tp \one\one^\tp \x }{n} - 1 + \frac{\bchi_i^\tp \one\one^\tp \bchi_i}{n}=-1,
\end{align*}
since $\one^\tp \x=\one^\tp \bchi_i=1$. 
\end{proof}

\begin{lemma}
\label{lem:alt}
For any non-empty $U\subset V$, 
\[a_U=\frac{n-|U|}{w(\delta^+ U)}c^+_{U^c},\quad \text{and}\quad a^+_U = \frac{n-|U|}{w(\delta U)} c_{U^c}.\]
\end{lemma}
\begin{proof}
This is a consequence of identities \eqref{eq:c(S_U)} and \eqref{eq:c(S+_U)} and Lemmas \ref{lem:alt_and_centroid} and \ref{lem:||alt||}. Applying the latter and then the former, observe that 
\[a_U = \frac{\norm{a_U}_2}{\norm{c^+_{U^c}}_2}c^+_{U^c} = \frac{\sqrt{w(\delta^+ U)}}{\sqrt{w(\delta^+U)}/|U^c|}c^+_{U^c} = \frac{n-|U|}{w(\delta^+U)}c^+_{U^c}.\]
A similar computation holds for $a_U^+$.  
\end{proof}


\begin{lemma}
Let $G=(V,E,w)$ be a weighted, undirected graph. Then 
\begin{equation*}
    \la c(\splx_{U_1}),c(\splx_{U_2})\ra = - \frac{w(\delta U_1\cap \delta U_2)}{|U_1||U_2|},\quad\text{and}\quad \la \alt^+_{U_1},\alt^+_{U_2}\ra = -\frac{w(\delta U_1^c\cap \delta U_2^c)}{w(\delta U_1)w(\delta U_2)}.
\end{equation*}
\end{lemma}
\begin{proof}
For $i,j\in V$, observe that $\chi_{U_1}^\tp \L_e\chi_{U_2}=-w(i,j)$. Therefore, 
\begin{align*}
    \la c_{U_1},c_{U_2}\ra &= \la |U_1|^{-1}\Sv\chi_{U_1},|U_2|^{-1}\Sv\chi_{U_2}\ra = |U_1|^{-1}|U_2|^{-1} \chi_{U_1}^\tp \L_G\chi_{U_2} \\
    &= |U_1|^{-1}|U_2|^{-1}\sum_{i\sim j} \chi_{U_1}^\tp \L_{(i,j)}\chi_{U_2} = |U_1|^{-1}|U_2|^{-1}\sum_{(i,j)\in \delta U_1\cap \delta U_2} -w(i,j),
\end{align*}
which proves the first equality. The second is shown similarly by employing Lemma \ref{lem:alt} and the previous identity:
\begin{align*}
    \la \alt^+_{U_1},\alt^+_{U_2}\ra &= \frac{|U_1^c||U_2^c|}{w(\delta U_1) w(\delta U_2)}\la c_{U_1^c},c_{U_2^c}\ra
    =-\frac{w(\delta U_1^c\cap \delta U_2^c)}{w(\delta U_1)w(\delta U_2)}.\qedhere
\end{align*}

\end{proof}

\section{Properties of \texorpdfstring{$\splxn_G$}{the normalized Simplex}}

Here we study the normalized simplex $\splxn_G$ of the graph $G$, a somewhat less accessible object than its unnormalized counterpart.  The normalized simplex is, roughly speaking, distorted by the weights of the vertices. Consequently, many of the relationships between $\splx_G$ and $\splx_G^+$ are lost between $\splxn_G$ and $\splxn_G^+$. The first issue is that, in general, $\splxn_G$ and its inverse are not centred at the origin. Indeed, recall that the zero eigenvector $\vpn_n$ of $\Ln_G$ sits in the space $\spn(\W^{1/2}_G\one)$, which is distinct from $\spn(\one)$ unless $\W^{1/2}_G=d\I$ for some $d$, in which case $G$ is regular.
If $G$ is not regular, we thus have that $\vp_i \in \spn(\W^{1/2}_G\one)\subset \spn(\one)^\perp$ for all $i<n$ implying that $\la \vp_i,\one\ra \neq 0$. In this case then,  
 \[\cent(\splxn_G) = \frac{1}{n} \Evaln^{1/2}\Eign^\tp \one = \frac{1}{n} \begin{pmatrix}
 \sqrt{\lambda_1}\la \vp_1,\one\ra \\
 \vdots \\
\sqrt{ \lambda_{n-1}}\la \vp_{n-1},\one \ra
 \end{pmatrix}\neq \zero.\]
The above argument proves the following.

\begin{lemma}
	\label{lem:centroid_normalized}
	The centroid of $\splxn_G$ coincides with the origin of $\R^{n-1}$ iff $G$ is regular. 
\end{lemma}

Given this, one might wonder whether the origin is even a point in the simplex $\splxn$. It is easily seen that it is, however. Consider the barycentric coordinate $\u = \sqrt{\w}/\norm{\sqrt{\w}}_1$, where $\sqrt{\w}=(w(1)^{1/2}, \dots, w(n)^{1/2})$. Since all eigenvectors $\vpn_i$, $i<n$ are orthogonal to $\vp_n \in\spn(\w^{1/2})$ it follows that $\zero = \Svn\u\in\splxn$. 

The next set of properties which don't hold between $\splxn$ and $\splxn^+$ are the orthogonality relationships present between a simplex and its dual. That is, in general $\splxn^+_G$ is the not the dual of $\splxn_G$. 

\begin{lemma}
	\label{lem:inverse_not_dual}
	The inverse simplex $\splxn_G^+$ is the dual of $\splxn_G$ iff $G$ is regular. 
\end{lemma}
\begin{proof}
	For any $i,j,k\in\N$ write 
	\begin{equation}
	\label{eq:lem_inverse_not_dual}
	\la \svn_i^+,\svn_j-\svn_k\ra = \delta_{ij} - \delta_{ik} + \frac{\sqrt{w(i)w(k)}}{n} - \frac{\sqrt{w(i)w(j)}}{n}.
	\end{equation}
	First suppose that $G$ is $k$-regular. Then for $i\neq k$, Equation \eqref{eq:lem_inverse_not_dual} becomes $\la \svn_i^+,\svn_j-\svn_k\ra = \delta{ij}$. Since $k$ was arbitrary, we see that $\{\svn_i^+\}$ is the sister pair of $\{\svn_j-\svn_k\}$. Conversely, suppose $G$ is not regular and let $i,k$ obey $0\neq w(i)\neq w(k)$. Taking $i=j\neq k$ in \eqref{eq:lem_inverse_not_dual} we see 
	\[\la \svn_i^+,\svn_i-\svn_k\ra = 1 - \frac{\sqrt{w(i)}}{n}(\sqrt{w(k)}-\sqrt{w(i)})\neq 1,\]
	so $\{\svn_i^+\}$ is not the sister set of $\{\svn_j-\svn_k\}$, completing the argument. 
\end{proof}


Recall from Section \ref{sec:background_spectral} that a subset of vertices is degree homogenous if each vertex in the set has the same weight. 

\begin{lemma}
	Let $U_1,U_2\subset V(G)$ be two non-empty, degree homogenous subsets such that $U_1\cap U_2=\emptyset$. Then the faces $\splxn[U_1]$ and $\splxn[U_2]$ are perpendicular. 
\end{lemma}
\begin{proof}
	Suppose $w(i)=w_1$ for all $i\in U_1$ and $w(i)=w_2$ for all $i\in U_2$. Let $\x_{U_i}$ be the barycentric coordinate of any point in $\splxn[U_i]$, $i=1,2$, and compute 
	\begin{align*}
	\la \Svn\x_{U_1},\Svn\x_{U_2}\ra &= \x_{U_1}^\tp \bigg(\I-\frac{\sqrt{\w}\sqrt{\w}^\tp}{n}\bigg) \x_{U_2}\\
	&= \x_{U_1}^\tp\x_{U_2} - \frac{1}{n}\sum_{i\in U_1}\x_{U_1}(i)\sqrt{w(i)}\sum_{j\in U_2}\x_{U_2}(j) \sqrt{w(j)} \\
	&= -\frac{1}{n}\sqrt{w_1w_2}\sum_{i\in U_1}\x_{U_1}(i) \sum_{j\in U_2}\x_{U_2}(j) = -\frac{\sqrt{w_1w_2}}{n},
	\end{align*} 
	where the second equality is due to fact that $U_1\cap U_2=\emptyset$. This demonstrates that $\la \Svn\x_{U_1},\p-\q\ra=0$ for any $\p,\q\in\splxn[U_2]$, completing the proof. 
\end{proof}

\paragraph{Centroids and Altitudes.}
For the normalized Laplacian we have
\begin{align}
\Lnf(\chi_U)&=\sum_{i\sim j}w(i,j) \bigg(\frac{\chi_U(i)}{\sqrt{w(i)}}-\frac{\chi_U(j)}{\sqrt{w(j)}}\bigg)^2 \notag \\
&= \sum_{i\in U,j\in U^c}w(i,j)\bigg(\frac{\chi_U(i)}{\sqrt{w(i)}}-\frac{\chi_U(j)}{\sqrt{w(j)}}\bigg)^2 \notag \\
&= \sum_{i\in U,j\in U^c}w(i,j)\frac{\chi_U(i)}{w(i)} \notag \\
&= \sum_{i\in U}\frac{W(\delta(i)\cap U^c)}{w(i)}. \label{eq:Lnf(chiU)}
\end{align}
Let $\gamma(i,B)$ denote the \emph{fractional weight of $i$ in $B$}; that is, \[\gamma(i,B)=\frac{1}{w(i)}\sum_{j\in\delta(i)\cap B} w(i,j)=\frac{1}{w(i)}W(\delta(i)\cap U^c),\]
and $\gamma(A,B)$ the \emph{average fractional weight from $A$ to $B$}: 
\[\gamma(A,B) =\frac{1}{|A|}\sum_{i\in A}\gamma(i,B). \]
Then, 
\begin{equation*}
\norm{c(\splxn_U)}_2^2 = \frac{1}{|U|^2}\la \Svn\chi_U,\Svn\chi_U\ra = \frac{1}{|U|^2}\Lnf(\chi_U) = \frac{1}{|U|} \gamma(U,U^c).
\end{equation*}


That is, the length of the centroid $c(\splxn_U)$ captures the total fraction of weight between $U$ and $U^c$. 

\paragraph{Centred  Simplex}
\note{Unclear whether this notion is actually useful yet. Unclear whether the centred version is easier to study than the unormalized  counterpart.}


When discussing general graphs, it will be useful to study a  translated copy of $\splxn_G$ which is centred at the origin. Accordingly, given any simplex $\ssplx$ with vertices $\{\sv_i\}$, we let $\ssplx^0$ denote the simplex with vertices $\{\sv_i - \cent(\ssplx)\}$. It's clear that the centroid of $\ssplx^0$ is the origin: 
\begin{align*}
\cent(\ssplx^0) 
&= \frac{1}{n}(\sv_1-\cent(\ssplx),\;\dots\;\sv_n-\cent(\ssplx))\one \\
&= \frac{1}{n}(\sv_1\;\dots\;\sv_n)\one - \frac{1}{n}(\cent(\ssplx)\;\dots\;\cent(\ssplx))\one = \cent(\ssplx) - \cent(\ssplx)=\zero.
\end{align*}

We solidify the concept with a definition. 

\begin{definition}
	Given a simplex $\ssplx$, the unique (up to rotation and translation) simplex with vertex matrix $\Sv(\ssplx) - (\cent(\ssplx)\;\dots\;\cent(\ssplx))$ centred at the origin is called the \emph{canonical (or centred) simplex corresponding to $\ssplx$} and is denoted $\ssplx^0$. 
\end{definition}

Noting that 
\begin{equation*}
\cent(\splxn) = \frac{1}{n}\bigg(\sum_{\ell=1}^n \svn_\ell(1),\dots,\sum_{\ell=1}^n \svn_\ell(n)\bigg)^\tp,
\end{equation*}
we see that the vertices of $\splxn_0$ have coordinates
\begin{equation*}
\svn_i(j) - \cent(\splxn)(j) = \vpn_j(i)\lambdan_j^{1/2} - \frac{1}{n}\sum_{\ell=1}^n \vpn_j(\ell)\lambdan_j^{1/2} = \lambdan_j^{1/2} \bigg(\vpn_j(i) - \frac{1}{n}\la \vpn_j,\one\ra \bigg).
\end{equation*}
Likewise, the vertices of $\splx_0^+$ have coordinates 
\begin{equation*}
\svn_i^+(j) = \evaln_j^{-1/2}\bigg(\vpn_j(i) - \frac{1}{n}\la \vpn_j,\one\ra\bigg).
\end{equation*}

One might imagine (and could be forgiven for doing so) that once centred, the normalized simplex and its inverse would indeed be duals. However, this is not the case. Observe that 
\begin{align*}
\la \cent(\splx^+),\svn_j-\svn_k\ra &= \frac{1}{n}\sum_{\ell=1}^{n-1} \sum_{r=1}^n \svn^+_r(\ell)(\svn_j(\ell)-\svn_k(\ell))  \\&= 
\frac{1}{n}\sum_{r=1}^n \sum_{\ell=1}^{n-1} (\vp_\ell(r)\vp_\ell(j) - \vp_\ell(r)\vp_\ell(k) ) \\
&= \frac{1}{n} \sum_{r=1}^n \bigg(\delta_{r,k} - \delta_{r,j} + \frac{\sqrt{w(r)w(k)}}{n} - \frac{\sqrt{w(r)w(j)}}{n}\bigg) \\
&= \frac{\sqrt{w(k)}-\sqrt{w(j)}}{n^2} \sum_{r\in[n]}\sqrt{w(r)},
\end{align*}
so, by Equation \eqref{eq:lem_inverse_not_dual},
\begin{align*}
\la \svn_i^+-\cent(\splxn^+),(\svn_j-\cent(\splx)) - (\svn_k-\cent(\splx))\ra &= \la \svn_i^+,\svn_j-\svn_k\ra - \la \cent(\splxn^+),\svn_j-\svn_k\ra \\
&= \delta_{ij} - \delta_{ik} + \frac{\sqrt{w(i)}}{n}(\sqrt{w(k)} - \sqrt{w(j)})  \\
&\qquad - \frac{\sqrt{w(k)}-\sqrt{w(j)}}{n^2} \sum_{r\in[n]}\sqrt{w(r)} \\
&= \delta_{ij} + (\sqrt{w(k)}-\sqrt{w(j)})\bigg(\frac{\sqrt{w(i)}}{n} - \frac{1}{n^2}\sum_{r\in[n]}\sqrt{w(r)}\bigg),
\end{align*}
which is not equal to $\delta_{ij}$ unless $w(k)=w(j)$ or $\sqrt{w(i)}= \frac{1}{n}\sum_r \sqrt{w(r)}$. Since this would have to hold for all $i,j,k$, $i,j\neq k$, both of these conditions imply that the graph would have to be regular. \note{So what IS the dual of these simplices??  Ughgggghghgh wtf is happening}







\section{Construction via Extended Menger and Gramian}
In this section we derive matrix equations which relate the geometry of hyperacute simplices and their duals. The equations appeal to the relationship between hyperacute simplices and graphs by using well known results from the literature on electrical networks and effective resistance. The goal of this section is to demonstrate to the reader the utility of the graph-simplex correspondence in generating statements about hyperacute simplices, by hijacking our knowledge of graph theory. 

Let a centred, hyperacute simplex $\splx^+$ be given. By Theorem \ref{thm:graph-simplex} it is the inverse simplex of a graph $G$, whose corresponding simplex $\splx = \splx_G$ is dual to $\splx^+$.
Therefore, $\L_G=\Sv^\tp\Sv$ and $\L_G^+=(\Sv^+)^\tp\Sv$ and the vertices $\sv_i^+$ of $\splx^+$ can be written as $\sv_i^+=(\vp_1(i)\lambda_1^{1/2},\dots,\vp_{n-1}(i)\lambda_{n-1}^{1/2})^\tp$. Hence, 

\begin{align*}
\norm{\sv_i^+-\sv_j^+}_2^2 = (\bchi_i-\bchi_j)^\tp (\Sv^+)^\tp\Sv^+ (\bchi_i-\bchi_j) = \effr(i,j).
\end{align*}

That is, the distance between the vertices of $\splx^+$ encodes the effective resistance between nodes $i$ and $j$ in $G$. Let $\D^+$ be the distance matrix of $\splx^+$ (thus the effective resistance matrix $\Reff_G$ of $G$). 


\note{Detour into electrical networks, not sure where this goes exactly.} Observe that starting with the equation $\Reff=\one\diag(\L_G^+(i,i))^\tp + \diag(\L_G^+(i,i))\one^\tp - 2\L_G^+$, \note{should explain where this equation comes from} it follows that $\x^\tp \Reff\x = -2\x^\tp \L_G^+\x$ for any $\x\in\spn(\one)^\perp$. Therefore, 
\begin{align*}
\L_G^+(i,j)&=\bchi_i^\tp \L_G^+\bchi_j \\
&= \bigg(\bchi_i-\frac{1}{n}\one\bigg)^\tp \L_G^+\bigg(\bchi_j-\frac{1}{n}\one\bigg) \\
&= -\frac{1}{2}\bigg(\bchi_i-\frac{1}{n}\one\bigg)^\tp \Reff_G\bigg(\bchi_j-\frac{1}{n}\one\bigg) \\
&= \frac{1}{2n}\bigg(\sum_{k\in[n]}\effr(i,k)+\effr(j,k)\bigg) - \frac{1}{2}\effr(i,j) -\frac{R_G}{n^2},
\end{align*}
where $R_G$ is the total effective resistance of the graph. 
For $i=j$, this becomes 
\begin{align*}
\L_G^+(i,i) = \frac{1}{n} \sum_{k\in[n]}\effr(i,k) - \frac{R_G}{n^2}.
\end{align*}

Let $\overline{d}$ be the average squared distance between all the vertices of $\splx^+$, that is
\[\overline{d} \equiv  \frac{1}{n^2}\sum_{i\leq j} \norm{\sv_i^+-\sv_j^+}_2^2.\]
Let $\xi(i)$ give the average squared distance of vertex $i$ from other vertices minus the total average distance, 
\[\xi(i) \equiv \frac{1}{n}\sum_{j} \norm{\sv_i^+-\sv_j^+}_2^2 - \overline{d},\]
and put $\bxi=(\xi(1),\dots,\xi(n))$. 
Then we have the following result. 

\begin{lemma}
	Let $\splx\subset\R^{n-1}$ be a centred hyperacute simplex with  squared distance matrix $\D$, and average squared distance vector $\bxi$. Denote by $\BGamma$ the vertex matrix of the dual simplex to $\splx$. Then, 
	\begin{equation}
	\label{eq:simplex_block_matrix}
	-\frac{1}{2} \begin{pmatrix}
	0 & \one_n^\tp \\ 
	\one_n &  \D
	\end{pmatrix}\begin{pmatrix}
	\bxi^\tp \BGamma^\tp \BGamma\bxi + 4\overline{d} & -(\BGamma^\tp \BGamma\bxi + 2\one/n)^\tp \\
	-(\BGamma^\tp \BGamma\bxi + 2\one/n) & \BGamma^\tp \BGamma
	\end{pmatrix} = \I_{n+1}.
	\end{equation}
	Moreover, the vertices of the dual simplex to $\splx$ and the distance matrix of $\splx$ are related by the equation
	\begin{equation}
	\label{eq:lem_inverse_relation}
	\BGamma^\tp \BGamma \D \BGamma^\tp \BGamma = -2\BGamma^\tp \BGamma,
	\end{equation}
	and in the space $\spn(\one)^\perp$ it holds that 
	\begin{equation*}
	\label{eq:lem_inverse_relation2}
	\D\BGamma^\tp \BGamma \D= -2\D.	
	\end{equation*}
\end{lemma}

\begin{proof}
As above, $\splx$ is the inverse simplex of some graph $G$, and therefore, $\D=\Reff$, where $\Reff$ is the effective resistance matrix. Therefore, we can rewrite $\xi(i)$ as 
\begin{align*}
\frac{1}{n}\sum_j \effr(i,j) - \frac{1}{n^2}\sum_{i<j}\effr(i,j),
\end{align*}
and $\bxi$ as 
\begin{align*}
\bxi = \frac{1}{n}\Reff\one - \frac{1}{n^2} \one \one^\tp \Reff\one = \frac{1}{n}\Reff\one - \frac{1}{n^2} \J \Reff\one.
\end{align*}
Meanwhile, the dual simplex to $\splx$ is the simplex of the graph $G$, and hence obeys $\BGamma^\tp\BGamma=\L_G$. Consequently, letting $\u=\frac{1}{n}\Reff\one - \frac{1}{n^2} \J \Reff\one$, we can rewrite Equation \ref{eq:simplex_block_matrix} as the purely graph theoretic statement  
\begin{equation*}
	-\frac{1}{2} \begin{pmatrix}
0 & \one_n^\tp \\ 
\one_n &  \Reff
\end{pmatrix} = 
\begin{pmatrix}
\u^\tp \L_G\u + \frac{4}{n^2}R & -(\L_G\u + \frac{2}{n}\one)^\tp \\
-(\L_G\u + \frac{2}{n}\one) & \L_G
\end{pmatrix}^{-1}.
\end{equation*}
where $R=\sum_{i<j}\effr(i,j)$ is the total effective resistance in the graph. The above equality was proved by Van Mieghem \etal~\cite{van2017pseudoinverse}, and in a more general form by Fiedler~\cite{fiedler1993geometric,fiedler2011matrices}, but we prove it here for completeness. Multiplying out the left hand side, the top left-hand corner of the resulting block matrix is
\[-\frac{1}{2}(\one^\tp\L_G - \frac{2}{n}\one^\tp \one) = 1,\]
since $\one^\tp \L_G=\one^\tp\L_G^\tp =\zero$. Likewise the top-right hand corner is $\zero$. The bottom left-hand corner is 
\begin{equation}
\label{eq:lem_block_matrix}
-\frac{1}{2}\bigg(\one\bxi^\tp \L_G\bxi +\frac{4}{n^2} R \one - \Reff\L_G\bxi - \frac{2}{n}\Reff\one\bigg),
\end{equation}
where, using that $\Reff=\bxi \one^\tp + \one \bxi^\tp -2\L_G^+$ and $\one^\tp\L_G=\zero$, 
\begin{align}
\Reff\L_G = \one\bxi^\tp \L_G - 2\bigg(\I-\frac{1}{n}\J\bigg).\label{eq:lem_block_matrix2}
\end{align}
Equation \eqref{eq:lem_block_matrix} thus becomes 
\begin{align*}
\frac{1}{n}\Reff\one -\frac{2}{n^2}R \one - \bigg(\I-\frac{1}{n}\J\bigg)\bxi &= \frac{1}{n}\Reff\one - \frac{2}{n^2}R\one - \bigg(\I-\frac{1}{n}\J\bigg)\bigg(\frac{1}{n}\Reff\one - \frac{1}{n^2}\J\Reff\one\bigg) \\
&= -\frac{2}{n^2} R\one + \frac{1}{n^2}\Reff\one + \frac{1}{n^2}\J\Reff\one - \frac{1}{n^3}\J^2\Reff\one \\
&= -\frac{2}{n^2} R\one +\frac{1}{n^2} \J\Reff\one = \zero,
\end{align*}
using that $\J^2 = n \J$, $R=\frac{1}{2}\one^\tp \Reff\one$, and $\J\Reff\one = \one(\one^\tp \Reff\one) = \one R$. Finally, again using \eqref{eq:lem_block_matrix2}, the bottom right-hand side is 
\begin{align*}
\frac{1}{2}\one \bxi^\tp \L_G + \frac{1}{n}\one\one^\tp - \frac{1}{2}\Reff\L_G &= \frac{1}{n}\J + \bigg(\I-\frac{1}{n}\J\bigg) = \I.
\end{align*}
This demonstrates that \eqref{eq:lem_block_matrix} holds. We now show that $\L_G \Reff\L_G = -2 \L_G$ and that $\Reff\L_G\Reff\x = -2\Reff\x$ for all $\x\in\spn(\one)^\perp$, which will complete the proof. Applying Equation \eqref{eq:lem_block_matrix2} we have 
\begin{align*}
\L_G\Reff\L_G = \L_G\one \bxi^\tp\L_G = -2\L_G+ \frac{2}{n}\L_G\one\one^\tp = -2\L_G.
\end{align*}
In the same way as \eqref{eq:lem_block_matrix2} was derived, we see that 
\begin{equation*}
\L_G\Reff = \L_G\bxi\one^\tp - 2\bigg(\I-\frac{1}{n}\J\bigg),
\end{equation*}
and so 
\begin{equation*}
\Reff\L_G\Reff= \bigg(\Reff\L_G\bxi^\tp + \frac{2}{n}\one \bigg)\one^\tp -2\Reff,
\end{equation*}
as desired. 
\end{proof}

Putting aside simplex geometry for the moment, it is worth meditating on the significance of Equation \eqref{eq:simplex_block_matrix} as applied to electrical networks. As demonstrated in \cite{van2017pseudoinverse}, the result translates into the matrix equation 

\begin{equation*}
-\frac{1}{2}\begin{pmatrix}
0 & \one^\tp \\
\one & \Reff
\end{pmatrix} = 
\begin{pmatrix}
\diag(\L_G^+(i,i))^\tp  \L_G \diag(\L_G^+(i,i)) 
&  -(\L_G\diag(\L_G^+(i,i)) + \frac{2}{n}\one )^\tp \\
-(\L_G\diag(\L_G^+(i,i)) + \frac{2}{n}\one ) 
& \L_G
\end{pmatrix}.
\end{equation*}

Let $\D$ be the distance matrix of a set $X$ of $d$ points. The matrix 
\begin{equation}
\label{eq:menger_matrix}
\begin{pmatrix}
0 & \one^\tp \\
\one & \D
\end{pmatrix}\in\R^{(d+1)\times (d+1)},
\end{equation}
is called the \emph{Menger matrix of $X$}. 

\begin{lemma}[\cite{menger1931new}]
	Let $\D$ be the distance matrix of a set $X$ of $d$ points. The $d-1$ dimensional volume of the convex hull of $X$ is proportional to the root of the determinant of the Menger matrix: 
	\begin{equation*}
	\vol(CH(X))^2 = \frac{(-1)^d}{((d-1)!)^2 2^{d-1}} \det\begin{pmatrix}
	0 & \one^\tp \\
	\one & \D
	\end{pmatrix}.
	\end{equation*} 
\end{lemma}

Sharpe~\cite{sharpe1967theorem} said something about something which should probably be cited, but not exactly sure what it is yet. 



\section{The Inverse Graph}
Let $G$ be a connected and weighted graph. $G$ admits the hyperacute combinatorial simplex $\splx_G$ which, by Theorem \ref{thm:graph-simplex}, is the inverse simplex of a graph $H$. It thus obeys 
\begin{equation*}
\norm{\sv_i-\sv_j}_2^2 = \effr_H(i,j). 
\end{equation*}
Expanding both sides for $i\neq j$ and  using the definition of the effective resistance yields 
\begin{equation*}
w_G(i) + w_G(j) + 2w_G(i,j) = \L_H^+(i,i) + \L_H^+(j,j) -2\L_H^+(i,j). 
\end{equation*}
Here we using the subscript $G$ to reinforce the fact that the weights are those of the original graph, $G$. We can calculate the entries of the pseudoinverse $\L_H^+$ more explicitly. Put 
\[W_G\equiv \sum_{i<j}w(i,j) = \frac{1}{2}\sum_i \sum_j w(i,j) = \frac{1}{2}\sum_i w(i).\]
That is, $W_G$ is the total weight of the graph $G$. Recall that $R_H$ is the total effective resistance of $H$ and compute 
\begin{align*}
R_H &= \sum_{i<j} \effr(i,j) \\
&= \sum_{i<j} (w_G(i) + w_G(j) + 2w_G(i,j)) \\
&= \frac{1}{2}\sum_{i,j} (w_G(i) +w_G(j)) + 2W_G \\
&= (2n+2)W_G 
\end{align*}

Using this and a previous calculated formula for the entries of the pseudoinverse yields 
\begin{align*}
\L_H^+(i,j) &= \frac{1}{2}\bigg(\sum_k \effr_H(i,k) + \sum_k \effr_H(j,k)\bigg) - \frac{1}{2} \effr_H(i,j) - \frac{R_H}{n^2}, \\
&= \frac{1}{2}\sum_k (w_G(i) + w_G(j) + 2w_G(k) + 2w_G(i,k) + 2w_G(j,k)) - \frac{1}{2}w_G(i,j) - \frac{R_H}{n^2}\\
&= \bigg(\frac{n}{2}+1\bigg)(w_G(i)+w_G(j))  - \frac{1}{2}w_G(i,j) + \bigg(2-\frac{2}{n}-\frac{2}{n^2}\bigg)W_G.
\end{align*}
where


\section{Inequalities}

The conductance of a graph $G$ is 
\begin{equation*}
\theta(S) \equiv \frac{|\delta(S)|}{|S|}. 
\end{equation*}
We have the following inequality: 
\[\theta(S)\geq \lambda_2\bigg(1-\frac{|S|}{|V|}\bigg)\geq \frac{\lambda_2}{2},\]
which yields 
\[\norm{\Sv \chi S}_2^2 \geq \frac{|S|}{2}\lambda_{n-1}.\]
We can relate the eigenvalues of $G$ to the geometry of $\splx$ via the relation $\Sv\Sv^\tp = \Eval$. Hence
\[\norm{\Sv \chi_S}_2^2 \geq \frac{|S|}{2} \Sv\Sv^\tp (n-1,n-1)\geq \frac{|S|}{2} \min_{i}\{(\Sv\Sv^\tp)(i,i):(\Sv\Sv^\tp)(i,i)\neq 0\} = \frac{|S|}{2}\min_{i=1}^{n-1} \norm{\Pi_i(\Sv)}_2^2.  \]


\begin{lemma}
If $\p$ is any vector pointing from $\splx_U$ to $\splx_{U^c}$ which has a non-empty intersection with both faces, then $\norm{\p}_2 \geq \norm{\alt(\splx_U)}_2$. 
\end{lemma}
\begin{proof}
Geometry. \note{Work this out}. 
\end{proof}

The following lemma is due to Devriendt and Van Mieghem~\cite{devriendt2018simplex}. 


\begin{lemma}
For any $f$ with $\la f,\one\ra=0$, \begin{equation*}
    \Lf(f) \geq \frac{\norm{f}_1^2}{4W(\delta^+F^+)},
\end{equation*}
for $F^+\equiv \{i:f(i)\geq 0\}$. 
\end{lemma}
\begin{proof}
Let $F^+$ be as above and let $F^-\equiv [n]\setminus F^+=\{i:f(i)<0\}$. Observe that 
\begin{equation*}
    \norm{f}_1=\sum_i |f(i)| = \la \chi_{F^+}-\chi_{F^-},f\ra = (\chi_{F^+}-\chi_{F^-})^\tp f = (\chi_{F^+}-\chi_{F^-})^\tp(\I-\J/n) f,
\end{equation*}
where the last inequality follows since $f$ is orthogonal to $\one$ by assumption. Using the pseudoinverse relation \eqref{eq:sv+sv}, we can continue as 
\begin{align*}
    \norm{f}_1 &= (\chi_{F^+}-\chi_{F^-})^\tp(\Sv^+)^\tp \Sv f \\
    &= (\chi_{F^+}-\one + \chi_{F^+})^\tp(\Sv^+)^\tp \Sv f \\
    &=2\chi_{F^+}^\tp (\Sv^+)^\tp \Sv f - (\Sv^+\one)^\tp \Sv f \\
    &= 2\la \Sv^+\chi_{F^+}, \chi_{F^+}^\tp (\Sv^+)^\tp \Sv f \ra &&\text{since }\Sv^+\one =\zero\\
    &\leq 2\norm{\Sv\chi_{F^+}}_2 \cdot \norm{\Sv^+ f}_2 &&\text{by Cauchy-Schwartz}\\
    &= 2\left(\chi_{F^+}\L^+\chi_{F^+}\cdot f^\tp \L f\right)^{1/2}.
\end{align*}
Squaring both sides and recalling that $\chi_{F^+}\L^+\chi_{F^+} = W(\delta^+F^+)$ gives the desired result. 
\end{proof}

We obtain several inequalities for the simplex via immediate application of inequalities from the literature on electrical networks. 

\note{Since $\Reff_G=n\sum_i\lambda_i^{-1}=n\tr(\Sv\Sv^\tp)$,
facts/inequalities pertaining to the effective resistance can be translated to the simplex. }

\begin{lemma}
Let $G=(V,E,w)$ be a weighted graph and let $U\subset V$ obey $\vol(U)<\vol(V)/2$ and \[\theta(U) \geq \frac{\alpha}{\vol(U)^{1/2-\eps}}.\]
\end{lemma}
\todo \note{finish this/decide whether this is worth including.}

\section{Steiner Circumscribed Ellipsoid}
\todo  \note{Read more about quadrics in general but filling this out. Might be more we can say. Also look into any algorithmic work done on quadrics. Does this relationship help us answer anything interesting?}
Fiedler derivation: ~\cite{fiedler2005geometry}. More Fiedler geometry: ~\cite{fiedler1993geometric}. 
A \emph{quadric} in $\C^d$ is a hypersurface of dimension $d-1$ of the form 
\begin{equation*}
\{\x\in\C^d: \x^\tp \Qb\x + \r^\tp \x + s=0\}.
\end{equation*}

\begin{definition}
\label{def:steiner_ellipsoid}
The \emph{Steiner Circumscribed Ellipsoid}, or simply the \emph{Steiner Ellipsoid} of a simplex $\splx$ with vertices $\{\sv_i\}$ is a quadric which contains the vertices and whose tangent plane at $\sv_i$ is parallel to the affine plane spanned by $\{\sv_j\}_{j\neq i}$. 
\end{definition}

\begin{theorem}
The Steiner ellipsoid of a simplex $\splx$ is unique and moreover, is the ellipsoid with minimum volume which contains $\splx$. 
\end{theorem}

Owing to its uniqueness, we denote the Steiner ellipsoid of the simplex $\splx$ by $\El(\splx)$. The following lemma gives an explicit representation of $\El(\splx)$. 

\begin{lemma}[\cite{fiedler2005geometry}]
The Steiner circumscribed Ellipsoid of $\splx=\splx(G)$ satisfies
\begin{equation}
\label{eq:steinerE}
    \El(\splx) = \bigg\{\x: \x^\tp \Sv^+(\Sv^+)^\tp \x - \frac{n-1}{n}=0\bigg\}.
\end{equation}
\end{lemma}
\begin{proof}
Set $\M=\Sv^+(\Sv^+)^\tp$ and $E=\{\x:\x^\tp \M\x=(n-1)/n\}$. The claim is that $\El(\splx)=E$.  
First we demonstrate that the vertices of $\splx$ are contained in $\El(\splx)$. Noticing that $\J^2=n\J$, we compute 
\begin{align*}
    \sv_i^\tp \M \sv_i &= \chi_i^\tp \Sv^\tp \Sv^+(\Sv^+)^\tp \Sv \chi_i = \chi_i^\tp \left(\I-\frac{1}{n}\J\right)^2 \chi_i = \chi_i^\tp \left(\I-\frac{1}{n}\J\right) \chi_i = 1 - \frac{1}{n}, 
\end{align*}
so indeed the vertices $\sv_i$ are contained in $E$. Now, define the hyperplane 
\[\H\equiv \bigg\{\x:\x^\tp \M \sv_i = -\frac{1}{n}\bigg\}.\]
We claim that $\H$ is the affine plane containing the points $\{\sv_j\}_{j\neq i}$. Indeed, consider $\sv_j$ for some fixed $j\neq i$. Then, as above 
\[\sv_j^\tp \M\sv_i = \chi_j^\tp \left(\I-\frac{1}{n}\J\right) \chi_i = -\frac{1}{n}. \]
It remains to show that $\H$ is parallel to the tangent plane of $E$ at the point $\sv_i$. But this tangent plane is defined by the equation~\cite{fiedler2005geometry} \note{Should figure out how this is actually done}
\[\x^\tp \M\sv_i =\frac{n-1}{n},\]
which is clearly parallel to $\H$. This completes the proof.
\end{proof}

Perhaps a more insightful representation of $\El(\splx)$ comes from appealing to Equation \eqref{eq:SvSv+}, i.e., $\Sv\Sv^\tp = \Eval^{-1/2}$. Hence, by \eqref{eq:steinerE},
\begin{equation}
\label{eq:steinerE2}
    \El(\splx) = \bigg\{\x:\x^\tp \Eval^{-1}\x = \frac{n-1}{n}\bigg\}.
\end{equation}







\section{Random Walks}
\note{Very unclear if there's anything interesting here. Mostly just contains Karel's thought on rws at the moment. Think about: \\
	
	(1) can we generate  a theory/answer questions regarding random walks in simplices using our knowledge of rws in graphs. \\
	
	(2)Straight lines are geodesics. If in the simplex the path created by a random walk is a straight line, is this telling us the random walk is as ``efficient'' as possible? Whereas those with curved lines are inefficient? Unclear how to formalized this /where to take it. 
	After  thinking about this a bit, probably not: It just says that the eigenvalues corresponding to  the vertices which contribute to the starting position are equal.
}



\subsection{Discrete Time Random Walks}
In a \emph{discrete time random walk (DSRW)} we envision a walker who jumps from vertex $i$ to vertex $j$ with probability proportional to $w(i,j)$. To this end, one defines the transition matrix 
\begin{equation*}
    \T(i,j) = \frac{w(i,j)}{w(i)}=\frac{\A_G(i,j)}{\sum_{k\in\delta(i)} \A_G(i,k)}.
\end{equation*}
It's clear that $\sum_i \T(i,j)=1$. 
The probability that the walker is at node $i$ at time $t$ is the probability that that she was at node $j$ at time $t-1$ and transitioned to node $i$. Thus, 
\begin{equation*}
    \pi_i(t)=\sum_{j}\pi_j(t-1)\T(i,j),
\end{equation*}
or, more succinctly, 
\begin{equation*}
    \bpi(t) = \T\bpi(t-1).
\end{equation*}
The stationary distribution $\bpi(\infty)\equiv \lim_t \bpi(t)$  satisfies $\bpi(\infty)=\T\bpi(\infty)$,  which yields that 
The stationary distribution of such a walk is given by 
\[\pi_i = \frac{\sum_{j\in \delta(i)} w(i,j)}{\sum_{j,k\in V}w(i,j)},\]
which, for an undirected and unweighted graph simplifies to $\pi_i=\deg(i)/2|E|$. 


\subsection{Continuous Time Random Walks}

A \emph{Continuous Time Random Walk}~\cite{masuda2017random} satisfies the equation 
\begin{equation}
\label{eq:ctrw_dynamics}
    \frac{d\bpi(t)}{dt}= -\bpi(t)^\tp\W^{-1} \L,
\end{equation}
hence 
\begin{equation*}
    \bpi(t)^\tp = \bpi(0)^\tp \exp(-\W^{-1}\L t).
\end{equation*}
After converging to the stationary distribution there is, by definition, no change in the distribution. Therefore, $d\bpi(t)/dt=0$ and Equation \eqref{eq:ctrw_dynamics} reduces to 
$-\bpi(t)\W^{-1}\L =\zero$. Therefore, $\bpi(t)\W^{-1}$ is a left eigenfunction of $\L$ or equivalently, $ \W^{-1}\bpi$ is a right eigenfunction with corresponding eigenvalue zero. Hence, $\W^{-1}\bpi\in\spn\{\one\}$, i.e., $\bpi\in\spn\{\w\}$. Since $\norm{\bpi(\infty)}_1=1$, we see that 
\[\bpi(\infty) = \frac{\w}{\norm{\w}_1}.\]
In particular, the CTRW shares the same stationary distribution as the DTRW. 



\subsection{mixing time}
The distribution $\bpi=(\pi_1,\dots,\pi_n)$ corresponds to a point in the simplex, namely $\p_\pi=\S\bpi$. It is thus natural to wonder whether this point tells us anything interesting about the dynamics of the walk. 

The \emph{variation distance} between two distributions $p_1$ and $p_2$ with finite state space $S$ is given by 
\[\norm{p_1-p_2}_V = \frac{1}{2}\sum_{s\in S} |p_1(s)-p_2(s)|.\]

\paragraph{Mixing Time.} Let $\p_i^t$ be the distribution over the set of vertices $V$ at time $t$ obtained by beginning the random walk at vertex $i$. Define 
\[\Delta(t) = \max_{i\in V}\norm{\p_i^t-\bpi}_V,\]
where $\norm{\cdot}_V$ is the variation distance. Given $\eps>0$ set 
\[\tau(\eps) = \min\{t:\Delta(t)\leq \eps\}.\]
We have 
