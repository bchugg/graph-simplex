\chapter{Introduction}


\section{Think about}
\begin{enumerate}		
	\item Been thinking about using the simplex as a means to sparsify the graph. But this is probably backwards. What about leveraging our knowledge vis-a-vis sparsifying graphs to ``sparsify'' a hyperacute simplex? Given simplex properties which can be expressed as a quadratic product, graph sparsification  techniques could  yield simplices with more orthogonality relations which maintain approximately the same properties. I suppose the question is whether a simplex with more orthogonality relationships is somehow easier to deal with? That is, why would it be advantageous to store a sparsified simplex?
	\item Can we use the simplex to bound eigenvalues?
	\item According to Gharan's notes, can optimize over $L_2^2$ metrics with SDPs. This should have implications for optimizing over the squared distances between vertices, which corresponds to optimizing over effective resistance. 
	\item In ~\cite{fiedler1998some}, Fielder gives some sort of correspondence involving ``ultrametric matrices''. Look this up and understand it---could be interesting. 
	\item Looking at the random walk of a graph as a path in the simplex didn't yield anything too interesting. What about the other way around? Beginning at a random point in the simplex, if we take a "random walk" (this would have to be defined appropriately -- we take a weighted step towards each vertex with some probability), we end up at some point that we know as a result of graph theory. We also know what governs how quickly we converge to this point, and when the path will be "straight". We know it's the sizes of the eigenvalues which govern the convergence; if we're simply given a hyperacute simplex, what do the eigenvalues represent? Can we translate this into a statement about the dynamics of the random walk in terms of the simplex only, and not the graph?
	\item Can we define the "inverse/dual" graph of $G$ as follows: $G$ yields a simplex $\splx_G$ which is hyperacute. It is therefore the inverse simplex of  graph $G^+$. How are $G$ and $G^+$ related? \note{Tried this in Section \ref{sec:inverse_graph}. Unclear as of yet whether it's interesting. }
	\item The projection matrix $Y(e,f)=b_e^\tp \L_G^+b_f\sqrt{w(e)w(f)}$ is symmetric with real eigenvalues (see \cite{vishnoi2013lx}). It thus yields a simplex. Maybe explore its properties. 
	\item Can use inequalities obtained in the effective resistance literature to obtain inequalities which pertain to all hyperacute simplices. See e.g.,\cite{alev2017graph} 
\item Do low rank approximations of the gram matrix maintain any of the simplex properties? This yields a smaller representation of the graph ... what properties does this representation have?
\item Embedding approximate distance matrix. 
\item Applications of Schur Complement? \note{try next}
    \item Simplex of the quotient graph? (EEP)

    \item Dimensionality reduction. Can we reduce the dimensionality in specific ways to maintain interesting properties? \note{Started thinking about this; JL lemma, sparsification, etc}
    \item Graph partitioning via the simplex? 
    \item Similarity measures between graphs. Projection onto different subspaces??
         
    \item We could use the correspondence to develop a theory of random simplices. This could be a useful model. Study the random geometry of simplices via this correspondence. The random model could simply be to consider a random graph $G(n,p)$ and look at its simplex. $p$ would roughl correspond to volume of the simplex --- higher $p$ implies higher connectivity implies larger volume. \note{Meeeeeeh. Not sure if interesting.}
    
    \section{Prior Work}
    
	Steinitz's theorem which investigates the relationship between undirected graphs arising from convex polyhedra in $\R^3$~\cite{steinitz1922polyeder}. 
    
    \section{Contribution}
   
\end{enumerate}
