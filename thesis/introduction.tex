\chapter{Introduction}
\label{chap:intro}
\chapterquote{Confusion is the natural  state of the mathematician.}{Lior Silberman}

\chapterquote{What if I slept a little more and forgot about all this nonsense.}{Franz Kafka, \emph{The Metamorphosis}}


This thesis is concerned with uniting two fundamental mathematical objects: the graph and the simplex. A graph is fundamentally a  \emph{combinatorial} object---it can be described purely by means of finite sets and must not refer to any underlying geometric space.  
Simplices, on  the other hand, are inherently  geometric. 
Essentially a high dimensional triangle, any complete description of a simplex must include certain geometric information; the distance between its vertices, for example. Thus, a simplex cannot be divorced from an  underlying metric space. 

The dubious reader may interject that graphs can \emph{of course} be viewed geometrically. For instance, he or she continues, it  is well-known that the shortest path between two vertices constitutes  a metric on  the graph. We in turn interrupt the interrupter and remark that while graphs \emph{can} be  given  geometric interpretations, it is \emph{not necessary} that  they are. Indeed, a graph can  be described by two finite lists: a list of its vertices and a second of the connections between these vertices (perhaps with  weights  given  to the  edges). No underlying geometric space need be defined. 

Due  to the combinatorial  nature of the graph and the geometric nature of the simplex, a connection between the two objects might seem unlikely a  priori. It is precisely this fact which makes such a connection worth studying. 
The original  link  between graphs and simplices was uncovered by 
Miroslav Fiedler  in his 1993 paper entitled ``A geometric approach to the Laplacian matrix of a graph"~\cite{fiedler1993geometric}. 
Here he introduced the machinery needed  to define what will  be  a central object in our study of the relationship between graphs  and simplices: a bijection between connected, weighted graphs and hyperacute simplices.
However, we will be concerned with more than this single bijective mapping. Indeed, what we will term the \emph{graph-simplex correspondence} includes four (not necessarily bijective) mappings between graphs and simplices. They arise as natural extensions  of  Fiedler's original work. 

Unfortunately (we believe) for  the mathematical community,  Fiedler's investigations in this area  have  gone relatively unnoticed. Convinced as we are of  the beauty and  utility of such  work, this dissertation aims to present Fiedler's  results in a concise, clarifying, and self-contained fashion,  expand on the mathematical  foundations of the correspondence, and explore new applications thereof. Our primary motivation is to convince the reader that the graph-simplex correspondence is a  useful tool for studying graphs and simplices,  and can shed light on various  aspects of both which are  overlooked by other  methods. Given the ubiquity of graphs in the mathematical  sciences, both in theory and in application, the possibility of a new tool with which to analyze them is highly  appealing. 



\section{Prior Work}
\label{sec:intro_prior_work}


As we stated  above, Miroslav Fiedler was the ``primary  mover''  in  uncovering the graph-simplex correspondence~\cite{fiedler1993geometric,fiedler2005geometry,fiedler2011matrices}. 
A lifelong geometer~\cite{vavvrin1995miroslav},  Fiedler made many contributions to both simplex geometry~\cite{fiedler1954geometry,fiedler1955geometry,fiedler1956geometry}, matrix theory~\cite{fiedler1998some,fiedler1995moore}, and graph theory~\cite{fiedler1973algebraic,  fiedler1975property, fiedler1976aggregation, fiedler1989laplacian}. 
However, his work connecting graph  theory and simplex geometry remained largely unnoticed until very recently, when Devriendt and Van Mieghem used the simplex geometry of the graph as intuition behind investigating  a graph's ``best conducting node''~\cite{van2017pseudoinverse} and, in a later work, provided a summary  of Fiedler's results~\cite{devriendt2018simplex}. All of this work is concerned  with a connected and possibly  weighted graph $G$ and what we will henceforth  refer to as its  \emph{combinatorial simplices},  denoted $\splx_G$ and $\splx_G^+$. (This is in contrast its \emph{normalized simplices}, which we  will define and explore later.) 

Fiedler uncovered  the graph-simplex correspondence by means of  a more general  relationship between matrices and simplices.  In particular, he associated with each symmetric matrix $\Q$ whose range space is orthogonal to the all ones vector (i.e.,  $\Q\one=\zero$) a unique (up to congruence) hyperacute simplex. Since the Laplacian  matrix  $\L_G$ of a connected, weighted graph  $G$ obeys this constraint, this associates with  each such graph a hyperacute simplex  $\splx_G^+$.   
For  reasons which will become clear later, we call $\splx_G^+$ the \emph{inverse} (combinatorial) simplex of $G$. 
Fiedler associated $\L_G$ and $\splx_G^+$  by means of a block matrix equation which involved several somewhat complex components, including  the Gram matrix of the outer normals of the simplex and  the radius of its circumscribed ellipsoid. While this matrix representation is useful for various reasons---elaborated upon in  Section~\ref{sec:block_matrix}---the correspondence can  be simplified  by means of working solely  with the graph's Laplacian  matrix. This is the approach  recently taken by Devriendt  and  Van  Mieghem~\cite{devriendt2018simplex}. They simplify and summarize Fiedler's main  results and focus mainly on one  side of the correspondence---namely,  given $G$,  they examine  the properties of  its associated  simplices $\splx_G$ and $\splx_G^+$. 
Devriendt and  Van Mieghem also make explicit the connection between a graph's (combinatorial) simplex, $\splx_G$, and  its  inverse  simplex, $\splx_G^+$. While Fiedler was aware  of the existence of  $\splx_G$---he later examines the properties of its circumscribed ellipsoid~\cite{fiedler2005geometry}---the  majority of his  work on  the graph-simplex  correspondence focuses on the inverse simplex, $\splx_G^+$. 

Due to Fiedler's more general interest in the relationship between matrices and simplices, the majority of  his  results pertaining  to the graph-simplex correspondence are implicit consequences thereof. His block matrix approach lends itself more readily to the study of volumes, angles and circumscribed quadrics, which thus constitute the core of Fiedler's results. Devriendt and Van Mieghem make many of these implicit results explicit, giving equations which directly relate properties of the  graph to those of the simplex. 
 
Very recently, the graph-simplex correspondence has been applied in  computer science to  the area of low dimensional graph embeddings. Such embeddings seek to realize the vertices of a given graph as points in Euclidean space  (ideally a space whose dimension is much  lower than the number of nodes of the  graph)  in such a way that particular graph properties are preserved~\cite{cai2018comprehensive}. 
Torres, Chan, and Eliassi-Rad examine
projections of $\splx_G$ into a lower dimensional space as possible graph  embeddings.~\cite{torres2019geometric}. They give empirical results suggesting that this approach is highly effective for link prediction  and graph  reconstruction. 

While this summarizes all  the work done explicitly on the  graph-simplex correspondence, the  more general topic of geometric graph theory has  garnered attention from many sources.  
There is a wide literature on graph embeddings and geometric graph visualizations (e.g.,~\cite{tamassia2013handbook,brass2007simultaneous,kamada1989algorithm,fruchterman1991graph,de1990draw}), an area  which typically seeks to represent a  graph (sometimes multiple graphs~\cite{erten2005simultaneous,evans2016simultaneous,blasius2012simultaneous}) in the plane or $\R^3$ under  certain  conditions. For example, we might seek  an embedding in which the edges do not cross (a ``planar'' embedding~\cite{kant1993algorithms,nishizeki2004planar}), or one in which the vertices are represented as geometric objects~\cite{dean1997rectangle}. 


Computer scientists have  also leveraged graph theory to analyze data. Datum with $k$  features can naturally be viewed  as points in $k$ dimensional space---the ``feature space''.  
Laplacian Eigenmap methods~\cite{belkin2002laplacian} assume that the observed data lies on a lower dimensional  manifold within the  feature space and seeks to develop useful representations of the data.  
A distinct approach involves trying to generate a lower dimensional representation of the  data given its graph structure  (typically represented as an ``affinity matrix''). This general approach is usually referred to as \emph{spectral embedding}~\cite{brand2003unifying,bengio2004learning}, and admits different instantiations,  
including \emph{Principal Component Analysis (PCA)}~\cite{jolliffe2011principal}, 
\emph{Multi-dimensional Scaling (MDS)}~\cite{kruskal1978multidimensional,cox2000multidimensional}, and 
\emph{Local  Linear Embedding (LLE)}~\cite{roweis2000nonlinear}. 
Related work seeks  to apply techniques from topology to  find structure in graphs, both from a purely theoretical viewpoint  (e.g., topological  graph theory~\cite{gross2001topological}),  and more recently with applications to complex networks in mind~\cite{salnikov2018simplicial,wu2015emergent}. 

There has also been work on graphs arising from general polyhedra, e.g., Steinitz's theorem~\cite{steinitz1922polyeder}. However, this work is not spectral in nature and therefore quite unrelated to the graph-simplex  correspondence. 

\section{Contribution}
\label{chap:intro_contribution}

	\begin{figure}
	\centering
	\begin{subfigure}[b]{0.32\textwidth}
		\centering
		\includegraphics[scale=0.3]{example_d=2_graph}
		\vspace{0.2cm}
		\subcaption{}
	\end{subfigure}
	\begin{subfigure}[b]{0.32\textwidth}
		\centering
		\includegraphics[scale=0.3]{example_d=2_comb}
		\subcaption{}
	\end{subfigure}
	\begin{subfigure}[b]{0.32\textwidth}
		\centering
		\includegraphics[scale=0.3]{example_d=2_norm}
		\subcaption{}
	\end{subfigure}
	\\ 
	\begin{subfigure}[b]{0.32\textwidth}
		\centering
		\includegraphics[scale=0.3]{example_d=3_graph}
		\vspace{0.2cm}
		\subcaption{}
	\end{subfigure}
	\begin{subfigure}[b]{0.32\textwidth}
		\centering
		\includegraphics[scale=0.5]{example_d=3_comb}
		\subcaption{}
	\end{subfigure}
	\begin{subfigure}[b]{0.32\textwidth}
		\centering
		\includegraphics[scale=0.5]{example_d=3_norm}
		\subcaption{}
	\end{subfigure}
	\caption{Two examples of graphs ((a) and (d)) and their combinatorial and normalized simplices.  The combinatorial simplices are figures (b) and (e); the red (lighter) simplex is the inverse combinatorial simplex. The normalized simplices are figures (c) and (f); the yellow (lighter) simplex is the inverse normalized simplex. Observe that the upper graph on three vertices gives rise to simplices in $\R^2$, while that on  four vertices to simplices in $\R^3$. The reader may notice that the  inverse simplex seems to  be smaller in volume---we will address this relationship in Chapter~\ref{chap:correspondence}.   }
	\label{fig:correspondence_examples}
\end{figure}



We provide a self-contained treatise of the graph-simplex correspondence, including both Fiedler's main results  on the topic as well those newly discovered results of Devriendt and Van Mieghem~\cite{devriendt2018simplex}. We also expand on these results in several ways,  enumerated  below. For a preliminary taste of the correspondence, see the examples  in Figure~\ref{fig:correspondence_examples}. 




\begin{itemize}
	\item {\bf Introduction of the dual simplex.} Although at first seemingly unrelated to the correspondence itself, we provide a  novel mathematical treatment of an object we call the ``dual simplex'' of a given simplex. This object was remarked upon by Fiedler in his 2011 book~\cite{fiedler2011matrices}, but he did not investigate it. We present several general properties of  the dual simplex (e.g., Lemmas~\ref{lem:dual_of_dual}, \ref{lem:dual_faces_orthogonal},  \ref{lem:hdesc_dual}, \ref{lem:dot_dual_vertices}) and use it to frame the graph-simplex correspondence, especially of the normalized Laplacian (see below). 
	\item {\bf Extension of correspondence to the normalized Laplacian.}  While Fiedler (implicitly) and Devriendt and Van Mieghem (explicitly) studied the correspondence by means of the combinatorial Laplacian of a graph, we expand the correspondence to the normalized Laplacian.  This matrix also describes the complete structure of the graph but is more intimately related to several of its features, such as random walk dynamics~\cite{chung1997spectral}. We introduce this new mapping  along with the original in Section~\ref{sec:bijection_graphs_simplices}. 
	We then study the properties of the simplex associated to the normalized Laplacian, which we term the ``normalized'' simplex.  
	Somewhat surprisingly, the normalized simplex is a significantly different object than the combinatorial  simplex. Its analysis  also proves more complicated because, as we will show, the inverse normalized simplex  is \emph{not} the dual of the normalized  simplex in general, whereas the combinatorial simplex and its inverse are duals to  one  another. 	 
	We refer  the reader to Figure~\ref{fig:correspondence} for an illustration of the relationship between a graph and its various simplices. 
	
	\begin{figure}
		\centering 
		\includegraphics[scale=0.4]{correspondence}
		\caption{An illustration of the various objects and relationships in the graph-simplex correspondence. The combinatorial simplices sit to the right of $G$, while the normalized simplices sit to the left. Duality  is marked by the superscript $\du$. 
			We see that  $\splx_G$ and $\splx_G^+$ are duals to one another but the normalized simplices are not.  }
		\label{fig:correspondence}
	\end{figure}
	

	\item {\bf New graph equations and inequalities.} 	Combining Fiedler's block matrix approach with that of Devriendt and Van Mieghem, we are able to uncover several new relationships. 
	We show, for  example,  that the entries of the Laplacian and the  vertices  of $\splx_G$ are related to the volumes of the facets of $\splx_G^+$ (Lemma~\ref{lem:L(i,i)_trees}), give a general formula for the volume of a simplex in terms of the eigenvalues of the Gram matrix of its dual  (Theorem~\ref{thm:simplex_volume}), and a formula for Steiner ellipsoid of a simplex in terms of the vertex matrix of its dual (Lemma~\ref{lem:El(T)_general}). 
	We also relate  the eigenvalues of $\Ln_G$  to the total weight of spanning trees in $G$, and  consequently  to the eigenvalues of $\L_G$ (Lemma~\ref{lem:prod_evalsn}). 
	These results are given  in  Section~\ref{sec:block_matrix}, \ref{sec:inequalities}, and~\ref{sec:quadrics}.  
	Figure~\ref{fig:property_triangle}  demonstrates how one can utilize the correspondence to translate between the combinatorial properties of the graph  and  the geometric properties of its  simplices. 
	 
	\item {\bf Link between $\re_G$ and $\splx_G^+$.} We uncover a link between the inverse combinatorial simplex of a graph and a geometric  object related to the effective resistance of the graph, which we call the ``resistive polytope'' and denote $\re_G$. It seems that the existence of this object has been previously acknowledged (e.g., ~\cite{shayanNotes}), but never rigorously studied. This material appears in Section~\ref{sec:resistive_polytope}. 
	
	\item {\bf Algorithmic analysis of the correspondence.} Perhaps most significantly, we initiate the study of the algorithmic foundations of the correspondence (Chapter~\ref{chap:algorithmics}). This entails three distinct aspects. 
	\begin{enumerate}
		\item {\bf Consequences for computational complexity.}	We explore several consequences for computational  complexity.  Owing to the pervasiveness of graphs in theory and application, the complexity class of many graph-theoretic problems are  well established (e.g., computing maximum-cuts and independent sets are ``hard'', while spanning trees are ``easy'', etc.) If, via the correspondence, such problems have analogues in the simplex then this has implications concerning the difficulty of these geometric problems. Moreover, while the complexity of  the analogous geometric problems may already  be known in general convex polytopes, understanding the complexity in simplices can yield an improved understanding of their hardness threshold. We give several examples of such results in Section~\ref{sec:algorithmics_complexity}. 
		\item {\bf Lower bounds on computing the correspondence.} We then explore the natural question of whether various aspects of the correspondence can be computed efficiently. For example, given $G$ how quickly can we compute $\splx_G$ or $\splx_G^+$? What about computing $\splx_G$ given $\splx_G^+$, or vice  versa? Our results in this  space are mostly negative; transitioning between many of these objects require time no less than that required to perform an eigendecomposition of a Laplacian matrix. 
		This is perhaps to be expected given that the mapping is based on such a decomposition, but it is not immediate.  It is a priori  feasible that the various relationships between the eigenvalues and eigenvectors which define the vertices of  the simplices are computable more quickly than the eigenvalues and eigenvectors  themselves. 
		\item {\bf Approximations.} Finally, we explore several approximations. Given that  the simplex of a graph with $n$ vertices lives in $\R^{n-1}$---a high dimensional  space---we might hope that we can ``approximately'' embed it  in  lower dimensions. We explore this possibility in Section~\ref{sec:algorithmics_JL}. We also demonstrate that rank $k$ approximations  to  the Laplacian give rise to convex polyhedra in $\R^k$, and that these polyhedra approximate the simplex $\splx_G$ in various ways (with the accuracy depending on the  size of the $(k+1)$-st  largest  eigenvalue of $\L_G$). We view these results as providing theoretical justification for recent work of Torres \etal mentioned  in  the previous section~\cite{torres2019geometric}.  
	\end{enumerate}

\end{itemize}



We  end this  section by noting that while the dissertation is largely theoretical  in nature, code implementing various  aspects of the  graph-simplex correspondence was written by the author and is publicly available~\cite{chugg2019graph}. The figures throughout the manuscript were either generated by this software (via \textsc{python} and \textsc{Matplotlib}),  or by the drawing editor \textsc{Ipe}~\cite{cheong2014ipe}.    

\begin{figure}
	\centering
	\includegraphics[scale=0.5]{property_triangle}
	\caption{A visualization of how the correspondence can be used to apply graph-theoretic knowledge to the geometry  of the simplices and vice versa. For example, leveraging that the geometry of $\splx_G^+$ is intimately related to the effective resistances of $G$ and relating the equations of $\splx_G^+$  to those of $\splx_G$ via duality allows us  to, say, express equations of spanning trees in terms of effective resistances.  }
	\label{fig:property_triangle}
\end{figure}


\section{Organization}
\label{sec:intro_organization}

The rest of the thesis will be  organized as follows. Chapter~\ref{chap:background} will present the relevant background material in the areas of linear algebra, spectral graph theory, and simplex  geometry. Here we will also define and make some preliminary explorations of the dual simplex. The  background material  of Sections~\ref{sec:background_general}, \ref{sec:background_linear},  and \ref{sec:background_laplacian} is quite standard; the reader familiar with  these subject areas should  be able to skip  them without too much trouble. We  encourage all readers to peruse Section~\ref{sec:background_simplices} because, for one, the field  of simplex geometry is  less  well  studied  in general than the others and secondly, as stated above, we provide a novel treatment of  the dual simplex. Chapters~\ref{chap:correspondence} and \ref{chap:further_properties} then explore the mathematical aspects  of  the graph-simplex correspondence, and Chapter~\ref{chap:algorithmics} presents the algorithmic foundations. In order  to conserve  space, we have moved those proofs which were   presented by either Fiedler or Devriendt and Van Mieghem to the appendix, in addition to those which are elementary and not directly related to the material at  hand (e.g., those pertaining  to background material). 










     

  	
